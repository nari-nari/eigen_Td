\section{等方弾性体の平衡方程式の一般解}

応力が$x_{3}$成分に依存しない場合の平衡方程式は以下のようになる.
\begin{align}
	\frac{\partial \sigma_{x_{1}x_{1}}}{\partial x_{1}}+\frac{\partial \sigma_{x_{1}x_{2}}}{\partial x_{2}}=0,\hspace{1cm}
	\frac{\partial \sigma_{x_{1}x_{2}}}{\partial x_{1}}+\frac{\partial \sigma_{x_{2}x_{2}}}{\partial x_{2}}=0
	\label{eq:2dequilibrium}
\end{align}
あるスカラー関数$\Phi$を用いて,応力テンソルの各成分を次式で表現することができれば,その応力は明らかに平衡方程式\eqref{eq:2dequilibrium}を満たす.
\begin{align}
	\sigma_{x_{1}x_{1}}=\frac{\partial^2 \Phi}{\partial x_{2}^2},\hspace{1cm}
	\sigma_{x_{2}x_{2}}=\frac{\partial^2 \Phi}{\partial x_{1}^2},\hspace{1cm}
	\sigma_{x_{1}x_{2}}=-\frac{\partial^2 \Phi}{\partial x_{1} \partial x_{2}}
	\label{eq:Airyxy}
\end{align}
このスカラー関数$\Phi$はThe Airy stress functionと呼ばれる.
弾性体を解析するには平衡条件に加えて適合条件を満たす必要がある.適合条件とは,
弾性体の複数の点の相対的な位置関係が変形によって変化しないことを保証する条件で,二次元の場合には歪みを用いて次式で与えられる.
\begin{align}
	\frac{\partial^2 e_{x_{1}x_{1}}}{\partial^2 x_{2}}
	-2\frac{\partial^2 e_{x_{1}x_{2}}}{\partial x_{1} \partial x_{2}}
	+\frac{\partial^2 e_{x_{2}x_{2}}}{\partial^2 x_{1}}
	\label{eq:Compatibility}
\end{align}
また,等方弾性体の場合,ひずみと応力の関係式は以下で表される.
\begin{align}
	\sigma_{ij}&=C_{ijkl}^{}e_{kl}^{}
	\nonumber
	\\
	C_{ijkl}&=\frac{3-\kappa}{\kappa-1}\mu\delta_{ij}\delta_{kl}
	+\mu(\delta_{ik}\delta_{jl}+\delta_{il}\delta_{jk})
	\label{eq:Constitute}
\end{align}
ただし,$\mu$は$\mu=\dsp\frac{E}{2(1+\nu)}$で与えられるラメ定数,
$\kappa$はKolosov's constantと呼ばれる定数で,
平面応力状態では式\eqref{eq:PlainStress},平面ひずみ状態では式\eqref{eq:PlainStrain}となる.
\begin{align}
	\kappa&=\frac{3-\nu}{1+\nu}\hspace{1cm}\text{(for plain stress)}
	\label{eq:PlainStress}
	\\
	\kappa&=3-4\nu\hspace{0.9cm}\text{(for plain strain)}
	\label{eq:PlainStrain}
\end{align}
ここで,式\eqref{eq:Airyxy}および\eqref{eq:Constitute}から,ひずみ$e_{ij}$を$\Phi$で表現すると次式のようになる.
\begin{align}
	e_{x_{1}x_{1}}=\frac{\kappa+1}{8\mu}\sigma_{x_{1}x_{1}}-\frac{3-\kappa}{8\mu}\sigma_{x_{2}x_{2}}
	=\frac{\kappa+1}{8\mu}\frac{\partial^2 \Phi}{\partial x_{2}^2}-\frac{3-\kappa}{8\mu}\frac{\partial^2 \Phi}{\partial x_{1}^2}
	\nonumber
	\\
	e_{x_{2}x_{2}}=\frac{\kappa+1}{8\mu}\sigma_{x_{2}x_{2}}-\frac{3-\kappa}{8\mu}\sigma_{x_{1}x_{1}}
	=\frac{\kappa+1}{8\mu}\frac{\partial^2 \Phi}{\partial x_{1}^2}-\frac{3-\kappa}{8\mu}\frac{\partial^2 \Phi}{\partial x_{2}^2}
	\nonumber
	\\
	e_{x_{1}x_{2}}=\hspace{0.3cm}\frac{1}{2\mu}\sigma_{x_{1}x_{2}}\hspace{0.3cm}
	=-\frac{1}{2\mu}\frac{\partial^2 \Phi}{\partial x_{1} \partial x_{2}}\hspace{1.5cm}
	\label{eq:DefPhi}
\end{align}
式\eqref{eq:Compatibility}に\eqref{eq:DefPhi}を代入し,整理すると次式が得られる.
\begin{align}
	\nabla^4\Phi=0
	\label{eq:biharmonic}
\end{align}
ゆえに$\Phi$は重調和関数となる.

二次元の極座標系において,式\eqref{eq:biharmonic}にはThe Michell solutionと呼ばれる一般解が存在し,$\Phi$は以下のように展開される.
\begin{align}
	\Phi =&A_{01}r^2+A_{02}r^2\ln(r)+A_{03}\ln(r)+A_{04}\theta
	\nonumber
	\\
	&+\Bigl(A_{11}r^3+A_{12}r\ln(r)+A_{14}r^{-1}\Bigr)\cos(\theta)+A_{13}r\theta\sin(\theta)
	\nonumber
	\\
	&+\Bigl(B_{11}r^3+B_{12}r\ln(r)+B_{14}r^{-1}\Bigr)\sin(\theta)+B_{13}r\theta\cos(\theta)
	\nonumber
	\\
	&+\sum_{m=2}^{\infty}\Bigl(A_{m1}r^{m+2}+A_{m2}r^{-m+2}
	+A_{m3}r^{m}+A_{m4}r^{-m}\Bigr)\cos(m\theta)
	\nonumber
	\\
	&+\sum_{m=2}^{\infty}\Bigl(B_{m1}r^{m+2}+B_{m2}r^{-m+2}
	+B_{m3}r^{m}+B_{m4}r^{-m}\Bigr)\sin(m\theta)
	\label{eq:Michell}
\end{align}

極座標表示における,応力テンソルの各成分$\sigma_{rr},\sigma_{r\theta},\sigma_{\theta\theta}$はThe Airy stress function $\Phi$を用いて次式で表現される.
\begin{align}
	\sigma_{rr}=\frac{1}{r}\frac{\partial \Phi}{\partial r}
	+\frac{1}{r^2}\frac{\partial^2 \Phi}{\partial \theta^2},\hspace{1cm}
	\sigma_{r\theta}=\frac{1}{r^2}\frac{\partial \Phi}{\partial \theta}
	-\frac{1}{r}\frac{\partial^2 \Phi}{\partial r\partial \theta},\hspace{1cm}
	\sigma_{\theta\theta}=\frac{\partial^2 \Phi}{\partial r^2}
	\label{eq:Airyrtheta}
\end{align}
ゆえに,式\eqref{eq:Airyrtheta}に式\eqref{eq:Michell}を代入すると,$\sigma_{rr},\sigma_{r\theta},\sigma_{\theta\theta}$は次式のようになる.
\begin{align}
	\sigma_{rr} =&2A_{01}
	+A_{02}\bigl\{ 2r\ln(r)+1\bigr\}
	+A_{03}r^{-2}
	\nonumber
	\\
	&+2A_{11}r\cos(\theta)
	+A_{12}r^{-1}\cos(\theta)
	+2A_{13}r^{-1}\cos(\theta)
	-2A_{14}r^{-3}\cos(\theta)
	\nonumber
	\\
	&+2B_{11}r\sin(\theta)
	+B_{12}r^{-1}\sin(\theta)
	-2B_{13}r^{-1}\sin(\theta)
	-2B_{14}r^{-3}\sin(\theta)
	\nonumber
	\\
	&+\sum_{m=2}^{\infty}\Bigl\{-A_{m1}(m+1)(m-2)r^{m}
	-A_{m2}(m+2)(m-1)r^{-m}
	\nonumber
	\\
	&\hspace{2.5cm}
	-A_{m3}m(m-1)r^{m-2}
	-A_{m4}m(m+1)r^{-m-2}\Bigr\}\cos(m\theta)
	\nonumber
	\\
	&+\sum_{m=2}^{\infty}\Bigl\{-B_{m1}(m+1)(m-2)r^{m}
	-B_{m2}(m+2)(m-1)r^{-m}
	\nonumber
	\\
	&\hspace{2.5cm}
	-B_{m3}m(m-1)r^{m-2}
	-B_{m4}m(m+1)r^{-m-2}\Bigr\}\sin(m\theta)
	\label{eq:SigmaRR}
\end{align}

\begin{align}
	\sigma_{r\theta} =&A_{04}r^{-2}
	\nonumber
	\\
	&+2A_{11}r\sin(\theta)
	+A_{12}r^{-1}\sin(\theta)
	-2A_{14}r^{-3}\sin(\theta)
	\nonumber
	\\
	&-2B_{11}r\cos(\theta)
	-B_{12}r^{-1}\cos(\theta)
	+2B_{14}r^{-3}\cos(\theta)
	\nonumber
	\\
	&+\sum_{m=2}^{\infty}\Bigl\{A_{m1}m(m+1)r^{m}
	-A_{m2}m(m-1)r^{-m}
	\nonumber
	\\
	&\hspace{1.5cm}
	+A_{m3}m(m-1)r^{m-2}
	-A_{m4}m(m+1)r^{-m-2}\Bigr\}\sin(m\theta)
	\nonumber
	\\
	&+\sum_{m=2}^{\infty}\Bigl\{-B_{m1}m(m+1)r^{m}
	+B_{m2}m(m-1)r^{-m}
	\nonumber
	\\
	&\hspace{1.5cm}
	-B_{m3}m(m-1)r^{m-2}
	+B_{m4}m(m+1)r^{-m-2}\Bigr\}\cos(m\theta)
	\label{eq:SigmaRTh}
\end{align}

\begin{align}
	\sigma_{\theta\theta} =&2A_{01}
	+A_{02}\bigl\{ 2r\ln(r)+3\bigr\}
	-A_{03}r^{-2}
	\nonumber
	\\
	&+6A_{11}r\cos(\theta)
	+A_{12}r^{-1}\cos(\theta)
	2A_{14}r^{-3}\cos(\theta)
	\nonumber
	\\
	&+6B_{11}r\sin(\theta)
	+B_{12}r^{-1}\sin(\theta)
	+2B_{14}r^{-3}\sin(\theta)
	\nonumber
	\\
	&+\sum_{m=2}^{\infty}\Bigl\{A_{m1}(m+1)(m+2)r^{m}
	+A_{m2}(m-1)(m-2)r^{-m}
	\nonumber
	\\
	&\hspace{2.0cm}
	+A_{m3}m(m-1)r^{m-2}
	+A_{m4}m(m+1)r^{-m-2}\Bigr\}\cos(m\theta)
	\nonumber
	\\
	&+\sum_{m=2}^{\infty}\Bigl\{B_{m1}(m+1)(m+2)r^{m}
	+B_{m2}(m-1)(m-2)r^{-m}
	\nonumber
	\\
	&\hspace{2.0cm}
	+B_{m3}m(m-1)r^{m-2}
	+B_{m4}m(m+1)r^{-m-2}\Bigr\}\sin(m\theta)
	\label{eq:SigmaThTh}
\end{align}
さらに極座標系の歪み-変位関係式は\eqref{eq:DefDisp}のようになる.
\begin{align}
	e_{rr}=\frac{\partial u_{r}}{\partial r},\hspace{1cm}
	e_{r\theta}=\frac{1}{2}\Bigl( \frac{1}{r}\frac{\partial u_{r}}{\partial \theta}
	+\frac{\partial u_{\theta}}{\partial r}-\frac{u_{\theta}}{r}\Bigr),\hspace{1cm}
	e_{\theta\theta}=\frac{1}{r}\frac{\partial u_{\theta}}{\partial \theta}+\frac{u_{r}}{r}
	\label{eq:DefDisp}
\end{align}
式\eqref{eq:SigmaRR}~式\eqref{eq:SigmaThTh},構成式\eqref{eq:Constitute},および歪み-変位関係式\eqref{eq:DefDisp}
を用いて$u_r$および$u_\theta$を求めると,以下のようになる.
\begin{align}
	2\mu u_{r} =&D_{1}\cos(\theta)+D_{2}\sin(\theta)
	\nonumber
	\\
	&+A_{01}(\kappa-1)r
	+A_{02}\Bigl\{ (\kappa-1)r\ln(r)-r\Bigr\}
	-A_{03}r^{-1}
	\nonumber
	\\
	&+A_{11}(\kappa-2)r^2\cos(\theta)
	+A_{12}\frac{1}{2}\Bigl\{ (\kappa+1)\theta\sin(\theta) -\cos(\theta)+(\kappa-1)\ln(r)\cos(\theta) \Bigr\}
	\nonumber
	\\
	&\hspace{0.5cm}+A_{13}\frac{1}{2}\Bigl\{ (\kappa-1)\theta\sin(\theta) -\cos(\theta)+(\kappa+1)\ln(r)\cos(\theta) \Bigr\}
	+A_{14}r^{-2}\cos(\theta)
	\nonumber
	\\
	&+B_{11}(\kappa-2)r^2\sin(\theta)
	+B_{12}\frac{1}{2}\Bigl\{ -(\kappa+1)\theta\cos(\theta) -\sin(\theta)+(\kappa-1)\ln(r)\sin(\theta) \Bigr\}
	\nonumber
	\\
	&\hspace{0.5cm}+B_{13}\frac{1}{2}\Bigl\{ (\kappa-1)\theta\cos(\theta) +\sin(\theta)-(\kappa+1)\ln(r)\sin(\theta) \Bigr\}
	+B_{14}r^{-2}\sin(\theta)
	\nonumber
	\\
	&+\sum_{m=2}^{\infty}\Bigl\{A_{m1}(\kappa-m-1)r^{m+1}
	+A_{m2}(\kappa+m-1)r^{-m+1}
	\nonumber
	\\
	&\hspace{3.8cm}
	-A_{m3}mr^{m-1}
	+A_{m4}mr^{-m-1}\Bigr\}\cos(m\theta)
	\nonumber
	\\
	&+\sum_{m=2}^{\infty}\Bigl\{B_{m1}(\kappa-m-1)r^{m+1}
	+B_{m2}(\kappa+m-1)r^{-m+1}
	\nonumber
	\\
	&\hspace{3.8cm}
	-B_{m3}mr^{m-1}
	+B_{m4}mr^{-m-1}\Bigr\}\sin(m\theta)
	\label{eq:UR}
\end{align}
\begin{align}
	2\mu u_{\theta} =&-D_{1}\sin(\theta)+D_{2}\cos(\theta)+D_{3}r
	\nonumber
	\\
	&+A_{02}(\kappa+1)r\theta
	-A_{04}r^{-1}
	\nonumber
	\\
	&+A_{11}(\kappa+2)r^2\sin(\theta)
	+A_{12}\frac{1}{2}\Bigl\{ (\kappa+1)\theta\cos(\theta) -\sin(\theta)-(\kappa-1)\ln(r)\sin(\theta) \Bigr\}
	\nonumber
	\\
	&\hspace{0.5cm}+A_{13}\frac{1}{2}\Bigl\{ (\kappa-1)\theta\cos(\theta) -\sin(\theta)-(\kappa+1)\ln(r)\sin(\theta) \Bigr\}
	+A_{14}r^{-2}\sin(\theta)
	\nonumber
	\\
	&-B_{11}(\kappa+2)r^2\cos(\theta)
	+B_{12}\frac{1}{2}\Bigl\{ (\kappa+1)\theta\sin(\theta) +\cos(\theta)+(\kappa-1)\ln(r)\cos(\theta) \Bigr\}
	\nonumber
	\\
	&\hspace{0.5cm}+B_{13}\frac{1}{2}\Bigl\{ -(\kappa-1)\theta\sin(\theta) -\cos(\theta)-(\kappa+1)\ln(r)\cos(\theta) \Bigr\}
	-B_{14}r^{-2}\cos(\theta)
	\nonumber
	\\
	&+\sum_{m=2}^{\infty}\Bigl\{A_{m1}(\kappa+m+1)r^{m+1}
	-A_{m2}(\kappa-m+1)r^{-m+1}
	\nonumber
	\\
	&\hspace{3.8cm}
	+A_{m3}mr^{m-1}
	-A_{m4}mr^{-m-1}\Bigr\}\sin(m\theta)
	\nonumber
	\\
	&+\sum_{m=2}^{\infty}\Bigl\{-B_{m1}(\kappa+m+1)r^{m+1}
	+B_{m2}(\kappa-m+1)r^{-m+1}
	\nonumber
	\\
	&\hspace{3.8cm}
	-B_{m3}mr^{m-1}
	-B_{m4}mr^{-m-1}\Bigr\}\cos(m\theta)
	\label{eq:UTh}
\end{align}
ただし,式\eqref{eq:UR}の$D_{1}\cos(\theta)+D_{2}\sin(\theta)$や
式\eqref{eq:UTh}の$-D_{1}\sin(\theta)+D_{2}\cos(\theta)+D_{3}r$は式\eqref{eq:DefDisp}を積分することで現れる不定積分の項である.
\newpage
