\section{等方弾性体の平衡方程式の一般解}

応力が$x_{3}$成分に依存しない場合の平衡方程式は以下のようになる.
\begin{align}
	\frac{\partial \sigma_{x_{1}x_{1}}}{\partial x_{1}}+\frac{\partial \sigma_{x_{1}x_{2}}}{\partial x_{2}}=0,\hspace{1cm}
	\frac{\partial \sigma_{x_{1}x_{2}}}{\partial x_{1}}+\frac{\partial \sigma_{x_{2}x_{2}}}{\partial x_{2}}=0
	\label{eq:2dequilibrium}
\end{align}
あるスカラー関数$\Phi$を用いて,応力テンソルの各成分を次式で表現することができれば,その応力は明らかに平衡方程式\eqref{eq:2dequilibrium}を満たす.
\begin{align}
\sigma_{x_{1}x_{1}}=\frac{\partial^2 \Phi}{\partial x_{2}^2},\hspace{1cm}
\sigma_{x_{2}x_{2}}=\frac{\partial^2 \Phi}{\partial x_{1}^2},\hspace{1cm}
\sigma_{x_{1}x_{2}}=-\frac{\partial^2 \Phi}{\partial x_{1}x_{2}}
\label{eq:Airyxy}
\end{align}
このスカラー関数$\Phi$はThe Airy stress functionと呼ばれる.さらに,等方弾性体の場合,$\Phi$は次の支配方程式を満たす重調和関数となる.
\begin{align}
\nabla^4\Phi=0
\label{eq:biharmonic}
\end{align}
二次元の極座標系において,式\eqref{eq:biharmonic}にはThe Michell solutionと呼ばれる一般解が存在し,$\Phi$は以下のように展開される.
\begin{align}
\Phi =&A_{01}r^2+A_{02}r^2\ln(r)+A_{03}\ln(r)+A_{04}\theta
\nonumber
\\
&+\Bigl(A_{11}r^3+A_{12}r\ln(r)+A_{14}r^{-1}\Bigr)\cos(\theta)+A_{13}r\theta\sin(\theta)
\nonumber
\\
&+\Bigl(B_{11}r^3+B_{12}r\ln(r)+B_{14}r^{-1}\Bigr)\sin(\theta)+B_{13}r\theta\cos(\theta)
\nonumber
\\
&+\sum_{m=2}^{\infty}\Bigl(A_{m1}r^{m+2}+A_{m2}r^{-m+2}
+A_{m3}r^{m}+A_{m4}r^{-m}\Bigr)\cos(m\theta)
\nonumber
\\
&+\sum_{m=2}^{\infty}\Bigl(B_{m1}r^{m+2}+B_{m2}r^{-m+2}
+B_{m3}r^{m}+B_{m4}r^{-m}\Bigr)\sin(m\theta)
\label{eq:Michell}
\end{align}

極座標表示における,応力テンソルの各成分$\sigma_{rr},\sigma_{r\theta},\sigma_{\theta\theta}$はThe Airy stress function $\Phi$を用いて次式で表現される.
\begin{align}
\sigma_{rr}=\frac{1}{r}\frac{\partial \Phi}{\partial r}
+\frac{1}{r^2}\frac{\partial^2 \Phi}{\partial \theta^2},\hspace{1cm}
\sigma_{r\theta}=\frac{1}{r^2}\frac{\partial \Phi}{\partial \theta}
-\frac{1}{r}\frac{\partial^2 \Phi}{\partial r\partial \theta},\hspace{1cm}
\sigma_{\theta\theta}=\frac{\partial^2 \Phi}{\partial r^2}
\label{eq:Airyrtheta}
\end{align}
ゆえに,式\eqref{eq:Airyrtheta}に式\eqref{eq:Michell}を代入すると,$\sigma_{rr},\sigma_{r\theta},\sigma_{\theta\theta}$は次式のようになる.
\begin{align}
\sigma_{rr} =&2A_{01}
+A_{02}\bigl\{ 2r\ln(r)+1\bigr\}
+A_{03}r^{-2}
\nonumber
\\
&+2A_{11}r\cos(\theta)
+A_{12}r^{-1}\cos(\theta)
+2A_{13}r^{-1}\cos(\theta)
-2A_{14}r^{-3}\cos(\theta)
\nonumber
\\
&+2B_{11}r\sin(\theta)
+B_{12}r^{-1}\sin(\theta)
-2B_{13}r^{-1}\sin(\theta)
-2B_{14}r^{-3}\sin(\theta)
\nonumber
\\
&+\sum_{m=2}^{\infty}\Bigl\{-A_{m1}(m+1)(m-2)r^{m}
-A_{m2}(m+2)(m-1)r^{-m}
\nonumber
\\
&-A_{m3}m(m-1)r^{m-2}
-A_{m4}m(m+1)r^{-m-2}\Bigr\}\cos(m\theta)
\nonumber
\\
&+\sum_{m=2}^{\infty}\Bigl\{-B_{m1}(m+1)(m-2)r^{m}
-B_{m2}(m+2)(m-1)r^{-m}
\nonumber
\\
&-B_{m3}m(m-1)r^{m-2}
-B_{m4}m(m+1)r^{-m-2}\Bigr\}\sin(m\theta)
\label{eq:SigmaRR}
\end{align}

\begin{align}
\sigma_{r\theta} =&A_{04}r^{-2}
\nonumber
\\
&+2A_{11}r\sin(\theta)
+A_{12}r^{-1}\sin(\theta)
-2A_{14}r^{-3}\sin(\theta)
\nonumber
\\
&-2B_{11}r\cos(\theta)
-B_{12}r^{-1}\cos(\theta)
+2B_{14}r^{-3}\cos(\theta)
\nonumber
\\
&+\sum_{m=2}^{\infty}\Bigl\{A_{m1}(m+1)(m-2)r^{m}
-A_{m2}(m+2)(m-1)r^{-m}
\nonumber
\\
&+A_{m3}m(m-1)r^{m-2}
-A_{m4}m(m+1)r^{-m-2}\Bigr\}\cos(m\theta)
\nonumber
\\
&+\sum_{m=2}^{\infty}\Bigl\{-B_{m1}(m+1)(m-2)r^{m}
+B_{m2}(m+2)(m-1)r^{-m}
\nonumber
\\
&-B_{m3}m(m-1)r^{m-2}
+B_{m4}m(m+1)r^{-m-2}\Bigr\}\sin(m\theta)
\label{eq:SigmaRTh}
\end{align}

\begin{align}
\sigma_{\theta\theta} =&2A_{01}
+A_{02}\bigl\{ 2r\ln(r)+3\bigr\}
-A_{03}r^{-2}
\nonumber
\\
&+6A_{11}r\cos(\theta)
+A_{12}r^{-1}\cos(\theta)
2A_{14}r^{-3}\cos(\theta)
\nonumber
\\
&+6B_{11}r\sin(\theta)
+B_{12}r^{-1}\sin(\theta)
+2B_{14}r^{-3}\sin(\theta)
\nonumber
\\
&+\sum_{m=2}^{\infty}\Bigl\{A_{m1}(m+1)(m+2)r^{m}
+A_{m2}(m-1)(m-2)r^{-m}
\nonumber
\\
&+A_{m3}m(m-1)r^{m-2}
+A_{m4}m(m+1)r^{-m-2}\Bigr\}\cos(m\theta)
\nonumber
\\
&+\sum_{m=2}^{\infty}\Bigl\{B_{m1}(m+1)(m+2)r^{m}
+B_{m2}(m-1)(m-2)r^{-m}
\nonumber
\\
&+B_{m3}m(m-1)r^{m-2}
+B_{m4}m(m+1)r^{-m-2}\Bigr\}\sin(m\theta)
\label{eq:SigmaThTh}
\end{align}
さらに構成式および歪み-変位関係式はそれぞれ\eqref{eq:constitute},\eqref{eq:DefDisp}のようになる.
\begin{align}
\sigma_{ij}&=C_{ijkl}e_{kl}
\nonumber
\\
C_{ijkl}&=\frac{3-\kappa}{2(\kappa-1)(1+\nu)}E\delta_{ij}\delta_{kl}+\frac{E}{1+\nu}\delta_{ik}\delta_{jl}
\label{eq:constitute}
\end{align}
\begin{align}
e_{rr}=\frac{\partial u_{r}}{\partial r},\hspace{1cm}
e_{r\theta}=\frac{1}{2}\Bigl( \frac{1}{r}\frac{\partial u_{r}}{\partial \theta}
+\frac{\partial u_{\theta}}{\partial r}-\frac{u_{\theta}}{r}\Bigr),\hspace{1cm}
e_{\theta\theta}=\frac{1}{r}\frac{\partial u_{\theta}}{\partial \theta}+\frac{u_{r}}{r}
\label{eq:DefDisp}
\end{align}

ただし,$\kappa$はKolosov's constantと呼ばれる定数で,平面応力状態では式\eqref{eq:PlainStress},平面ひずみ状態では式\eqref{eq:PlainStrain}となる.
\begin{align}
	\kappa&=\frac{3-\nu}{1+\nu}\hspace{1cm}\text{(for plain stress)}
	\label{eq:PlainStress}
	\\
	&=3-4\nu\hspace{1cm}\text{(for plain strain)}
	\label{eq:PlainStrain}
\end{align}


式\eqref{eq:SigmaRR}~\eqref{eq:DefDisp}を用いて$u_r$および$u_\theta$を求めると,以下のようになる.

\begin{align}
2\mu u_{r} =&D_{1}\cos(\theta)+D_{2}\sin(\theta)
\nonumber
\\
&+A_{01}(\kappa-1)r
+A_{02}\bigl\{ (\kappa-1)r\ln(r)-r\bigr\}
-A_{03}r^{-1}
\nonumber
\\
&+A_{11}(\kappa-2)r^2\cos(\theta)
+A_{12}\frac{1}{2}\bigl\{ (\kappa+1)\theta\sin(\theta) -\cos(\theta)+(\kappa-1)\ln(r)\cos(\theta) \bigr\}
\nonumber
\\
&\hspace{\fill}+A_{13}\frac{1}{2}\bigl\{ (\kappa-1)\theta\sin(\theta) -\cos(\theta)+(\kappa+1)\ln(r)\cos(\theta) \bigr\}
+A_{14}r^{-2}\cos(\theta)
\nonumber
\\
&+B_{11}(\kappa-2)r^2\sin(\theta)
+B_{12}\frac{1}{2}\bigl\{ -(\kappa+1)\theta\cos(\theta) -\sin(\theta)+(\kappa-1)\ln(r)\sin(\theta) \bigr\}
\nonumber
\\
&+B_{13}\frac{1}{2}\bigl\{ (\kappa-1)\theta\cos(\theta) +\sin(\theta)-(\kappa+1)\ln(r)\sin(\theta) \bigr\}
+B_{14}r^{-2}\sin(\theta)
\nonumber
\\
&+\sum_{m=2}^{\infty}\Bigl\{A_{m1}(\kappa-m-1)r^{m+1}
+A_{m2}(\kappa+m-1)r^{-m+1}
\nonumber
\\
&-A_{m3}mr^{m-1}
+A_{m4}mr^{-m-1}\Bigr\}\cos(m\theta)
\nonumber
\\
&+\sum_{m=2}^{\infty}\Bigl\{B_{m1}(\kappa-m-1)r^{m+1}
+B_{m2}(\kappa+m-1)r^{-m+1}
\nonumber
\\
&-B_{m3}mr^{m-1}
+B_{m4}mr^{-m-1}\Bigr\}\sin(m\theta)
\label{eq:UR}
\end{align}


\begin{align}
2\mu u_{\theta} =&-D_{1}\sin(\theta)+D_{2}\cos(\theta)+D_{3}r
\nonumber
\\
&+A_{02}(\kappa+1)r\theta
-A_{04}r^{-1}
\nonumber
\\
&+A_{11}(\kappa+2)r^2\sin(\theta)
+A_{12}\frac{1}{2}\bigl\{ (\kappa+1)\theta\cos(\theta) -\sin(\theta)-(\kappa-1)\ln(r)\sin(\theta) \bigr\}
\nonumber
\\
&\hspace{\fill}+A_{13}\frac{1}{2}\bigl\{ (\kappa-1)\theta\cos(\theta) -\sin(\theta)-(\kappa+1)\ln(r)\sin(\theta) \bigr\}
+A_{14}r^{-2}\sin(\theta)
\nonumber
\\
&-B_{11}(\kappa+2)r^2\cos(\theta)
+B_{12}\frac{1}{2}\bigl\{ (\kappa+1)\theta\sin(\theta) +\cos(\theta)+(\kappa-1)\ln(r)\cos(\theta) \bigr\}
\nonumber
\\
&+B_{13}\frac{1}{2}\bigl\{ -(\kappa-1)\theta\sin(\theta) -\cos(\theta)-(\kappa+1)\ln(r)\cos(\theta) \bigr\}
-B_{14}r^{-2}\cos(\theta)
\nonumber
\\
&+\sum_{m=2}^{\infty}\Bigl\{A_{m1}(\kappa+m+1)r^{m+1}
-A_{m2}(\kappa-m+1)r^{-m+1}
\nonumber
\\
&+A_{m3}mr^{m-1}
-A_{m4}mr^{-m-1}\Bigr\}\sin(m\theta)
\nonumber
\\
&+\sum_{m=2}^{\infty}\Bigl\{-B_{m1}(\kappa+m+1)r^{m+1}
+B_{m2}(\kappa-m+1)r^{-m+1}
\nonumber
\\
&-B_{m3}mr^{m-1}
-B_{m4}mr^{-m-1}\Bigr\}\cos(m\theta)
\label{eq:UTh}
\end{align}

\newpage
