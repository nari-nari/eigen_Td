\section{The Michell solutionの係数に関する連立方程式}

任意の$0\leq\pi\leq2\pi$に対して\eqref{eq:UR1eq}~\eqref{eq:SigmaRTh1eq}が成立するためには
$1$,$\cos(m\theta)$,$\sin(m\theta)$$(m\geq1)$の係数が0である必要がある.ゆえに,次下の連立方程式が得られる.

1の係数:
\begin{align}
	\left(
	\begin{array}{cccc}
		{\dsp\frac{1}{2\mu^b}}& 0 & 0 & {\dsp\frac{1}{2\mu^a}} \\
		0 & {\dsp\frac{\kappa^b-1}{2\mu^b}} & {\dsp\frac{1}{2\mu^a}} & 0 \\
		0 & 4 & -2 & 0 \\
		0 & 0 & 0 & -2 \\
	\end{array}
	\right)
	\left(
	\begin{array}{c}
		D_{3}^{(I-)} \\
	 	A_{01}^{(I-)} \\
		A_{03}^{(I+)}\\
	 	A_{04}^{(I+)} \\
	\end{array}
	\right)
	=
	\left(
	\begin{array}{c}
		0 \\
	 	0 \\
		0 \\
	 	0 \\
	\end{array}
	\right)
\end{align}
$\cos(\theta),\sin(\theta)$の係数:
\begin{align}
	\left(
	\begin{array}{ccc}
		{\dsp\frac{1}{2\mu^b}}& {\dsp\frac{\kappa^b-2}{2\mu^b}} & {\dsp-\frac{1}{2\mu^a}} \\
		{\dsp-\frac{1}{2\mu^b}} & {\dsp\frac{\kappa^b+2}{2\mu^b}} & {\dsp-\frac{1}{2\mu^a}} \\
		0 & 4 & 4 \\
	\end{array}
	\right)
	\left(
	\begin{array}{c}
		D_{1}^{(I-)} \\
	 	A_{11}^{(I-)} \\
	 	A_{14}^{(I+)} \\
	\end{array}
	\right)
	=
	\left(
	\begin{array}{c}
		u_{x}(\bm{x_{0}}) \\
	 	-u_{x}(\bm{x_{0}}) \\
	 	0 \\
	\end{array}
	\right)
\end{align}
\begin{align}
	\left(
	\begin{array}{ccc}
		{\dsp\frac{1}{2\mu^b}}& {\dsp\frac{\kappa^b-2}{2\mu^b}} & {\dsp-\frac{1}{2\mu^a}} \\
		{\dsp\frac{1}{2\mu^b}} & {\dsp-\frac{\kappa^b+2}{2\mu^b}} & {\dsp\frac{1}{2\mu^a}} \\
		0 & -4 & -4 \\
	\end{array}
	\right)
	\left(
	\begin{array}{c}
		D_{2}^{(I-)} \\
	 	B_{11}^{(I-)} \\
	 	B_{14}^{(I+)} \\
	\end{array}
	\right)
	=
	\left(
	\begin{array}{c}
		u_{y}(\bm{x_{0}}) \\
	 	u_{y}(\bm{x_{0}}) \\
	 	0 \\
	\end{array}
	\right)
\end{align}
\newpage
$\cos(m\theta),\sin(m\theta)$の係数$(m\geq2)$:
\begin{align}
	\left(
	\begin{array}{cccc}
		{\dsp\frac{ \kappa^b-m-1}{2\mu^{b}}}& {\dsp-\frac{m}{2\mu^{b}}} &
		{\dsp-\frac{\kappa^a+m-1}{2\mu^{a}}} & {\dsp-\frac{m}{2\mu^{a}}} \\
		{\dsp\frac{\kappa^b+m+1}{2\mu^{b}}} & {\dsp\frac{m}{2\mu^{b}}} &
		{\dsp\frac{\kappa^a-m+1}{2\mu^{a}}} & {\dsp-\frac{m}{2\mu^{a}}} \\
		-(m+1)(m-2) & -m(m-1) & (m+2)(m-1) & m(m+1) \\
		m(m+1) & m(m-1) & m(m-1) & m(m+1)\\
	\end{array}
	\right)
	\left(
	\begin{array}{c}
		A_{m1}^{(I-)} \\
	 	A_{m3}^{(I-)} \\
		A_{m2}^{(I+)}\\
	 	A_{m4}^{(I+)} \\
	\end{array}
	\right)
	=
	\left(
	\begin{array}{c}
		0 \\
	 	0 \\
		0 \\
	 	0 \\
	\end{array}
	\right)
\end{align}
\begin{align}
	\left(
	\begin{array}{cccc}
		{\dsp\frac{ \kappa^b-m-1}{2\mu^{b}}}& {\dsp-\frac{m}{2\mu^{b}}} &
		{\dsp-\frac{\kappa^a+m-1}{2\mu^{a}}} & {\dsp-\frac{m}{2\mu^{a}}} \\
		{\dsp-\frac{\kappa^b+m+1}{2\mu^{b}}} & {\dsp-\frac{m}{2\mu^{b}}} &
		{\dsp-\frac{\kappa^a-m+1}{2\mu^{a}}} & {\dsp \frac{m}{2\mu^{a}}} \\
		-(m+1)(m-2) & -m(m-1) & (m+2)(m-1) & m(m+1) \\
		-m(m+1) & -m(m-1) & -m(m-1) & -m(m+1)\\
	\end{array}
	\right)
	\left(
	\begin{array}{c}
		B_{m1}^{(I-)} \\
	 	B_{m3}^{(I-)} \\
		B_{m2}^{(I+)}\\
	 	B_{m4}^{(I+)} \\
	\end{array}
	\right)
	=
	\left(
	\begin{array}{c}
		0 \\
	 	0 \\
		0 \\
	 	0 \\
	\end{array}
	\right)
\end{align}
これらの連立方程式を解くと,
\begin{align}
	D_{1}^{(I-)}=2\mu^{b}u_{x}(\bm{x_{0}}),\hspace{0.5cm}
	D_{2}^{(I-)}=2\mu^{b}u_{y}(\bm{x_{0}}),\hspace{0.5cm}D_{3}^{(I-)}=0
	\nonumber
	\\
	A_{ij}^{(I-)}=0,\hspace{0.5cm}A_{ij}^{(I+)}=0,\hspace{0.5cm}
	B_{ij}^{(I-)}=0,\hspace{0.5cm}B_{ij}^{(I+)}=0
\end{align}
以上から,$\hat{\bm{w}}^{(I)}$は以下のようになる.
\begin{align}
	\hat{w}_{r}^{(I-)}=u_{x}(\bm{x_{0}})\cos(\theta)+u_{y}(\bm{x_{0}})\sin(\theta)
	\nonumber
	\\
	\hat{w}_{\theta}^{(I-)}=-u_{x}(\bm{x_{0}})\sin(\theta)+u_{y}(\bm{x_{0}})\cos(\theta)
	\nonumber
	\\
	\hat{w}_{r}^{(I+)}=0,\hspace{0.5cm}\hat{w}_{\theta}^{(I+)}=0
\end{align}

\newpage
