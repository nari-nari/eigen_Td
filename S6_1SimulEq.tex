\section{三角関数の係数に関する連立方程式}

任意の$0\leq\pi\leq2\pi$に対して\eqref{eq:UR1eq}~\eqref{eq:SigmaRTh1eq}が成立するためには$1$,$\theta\cos(\theta)$,$\theta\sin(\theta)$,
$\cos(m\theta)$,$\sin(m\theta)$$(m\geq1)$の係数が0である必要がある.ゆえに,次式の連立方程式が得られる.

1の係数:
\begin{align}
	\left(
	\begin{array}{cccc}
		1& 0 & 0 & 1 \\
		0 & \kappa^b-1 & 1 & 0 \\
		0 & 4\mu^{b} & -2\mu^{a} & 0 \\
		0 & 0 & 0 & -2\mu^{a} \\
	\end{array}
	\right)
	\left(
	\begin{array}{c}
		D_{3}^{*(I-)} \\
	 	A_{01}^{*(I-)} \\
		A_{03}^{*(I+)}\\
	 	A_{04}^{*(I+)} \\
	\end{array}
	\right)
	=
	\left(
	\begin{array}{c}
		0 \\
	 	0 \\
		0 \\
	 	0 \\
	\end{array}
	\right)
\end{align}
$\theta\cos(\theta),\theta\sin(\theta)$の係数:
\begin{align}
	\left(
	\begin{array}{c}
		A_{13}^{*(I-)} \\
	 	B_{13}^{*(I-)} \\
	\end{array}
	\right)
	=
	\left(
	\begin{array}{c}
		0 \\
	 	0 \\
	\end{array}
	\right)
\end{align}
$\cos(\theta),\sin(\theta)$の係数:
\begin{align}
	\left(
	\begin{array}{cccc}
		1& \kappa^b-2 & 1/(\kappa^b-1) & -1 \\
		-1 & \kappa^b+2 & 1/(\kappa^b-1) & -1 \\
		0 & 4\mu^{b} & 2\mu^{b}(\kappa^b-3)/(\kappa^b-1) & 4\mu^{a} \\
		0 & 4\mu^{b} & -2\mu^{b}(\kappa^b+1)/(\kappa^b-1) & 4\mu^{a} \\
	\end{array}
	\right)
	\left(
	\begin{array}{c}
		D_{1}^{*(I-)} \\
	 	A_{11}^{*(I-)} \\
		A_{13}^{*(I-)}\\
	 	A_{14}^{*(I+)} \\
	\end{array}
	\right)
	=
	\left(
	\begin{array}{c}
		u_{x}(\bm{x_{0}}) \\
	 	-u_{x}(\bm{x_{0}}) \\
		0 \\
	 	0 \\
	\end{array}
	\right)
\end{align}
\begin{align}
	\left(
	\begin{array}{cccc}
		1& \kappa^b-2 & -1/(\kappa^b-1) & -1 \\
		1 & -(\kappa^b+2) & 1/(\kappa^b-1) & 1 \\
		0 & 4\mu^{b} & -2\mu^{b}(\kappa^b-3)/(\kappa^b-1) & 4\mu^{a} \\
		0 & -4\mu^{b} & -2\mu^{b}(\kappa^b+1)/(\kappa^b-1) & -4\mu^{a} \\
	\end{array}
	\right)
	\left(
	\begin{array}{c}
		D_{2}^{*(I-)} \\
	 	B_{11}^{*(I-)} \\
		B_{13}^{*(I-)}\\
	 	B_{14}^{*(I+)} \\
	\end{array}
	\right)
	=
	\left(
	\begin{array}{c}
		u_{y}(\bm{x_{0}}) \\
	 	u_{y}(\bm{x_{0}}) \\
		0 \\
	 	0 \\
	\end{array}
	\right)
\end{align}
\newpage
$\cos(m\theta),\sin(m\theta)$の係数$(m\geq2)$:
\begin{align}
	\left(
	\begin{array}{cccc}
		\kappa^b-m-1& -m & -(\kappa^a+m-1) & -m \\
		\kappa^b+m+1 & m & \kappa^a-m+1 & -m \\
		-2\mu^{b}(m+1)(m-2) & -2\mu^{b}m(m-1) &
		2\mu^{a}(m+2)(m-1) & 2\mu^{a}m(m+1) \\
		2\mu^{b}m(m+1) & 2\mu^{b}m(m-1) &
		2\mu^{a}m(m-1) & 2\mu^{a}m(m+1) \\
	\end{array}
	\right)
	\left(
	\begin{array}{c}
		A_{m1}^{*(I-)} \\
	 	A_{m3}^{*(I-)} \\
		A_{m2}^{*(I+)}\\
	 	A_{m4}^{*(I+)} \\
	\end{array}
	\right)
	\nonumber
	\\
	=
	\left(
	\begin{array}{c}
		0 \\
	 	0 \\
		0 \\
	 	0 \\
	\end{array}
	\right)
\end{align}
\begin{align}
	\left(
	\begin{array}{cccc}
		\kappa^b-m-1& -m & -(\kappa^a+m-1) & -m \\
		-(\kappa^b+m+1) & -m & -(\kappa^a-m+1) & m \\
		-2\mu^{b}(m+1)(m-2) & -2\mu^{b}m(m-1) &
		2\mu^{a}(m+2)(m-1) & 2\mu^{a}m(m+1) \\
		-2\mu^{b}m(m+1) & -2\mu^{b}m(m-1) &
		-2\mu^{a}m(m-1) & -2\mu^{a}m(m+1) \\
	\end{array}
	\right)
	\left(
	\begin{array}{c}
		B_{m1}^{*(I-)} \\
	 	B_{m3}^{*(I-)} \\
		B_{m2}^{*(I+)}\\
	 	B_{m4}^{*(I+)} \\
	\end{array}
	\right)
	\nonumber
	\\
	=
	\left(
	\begin{array}{c}
		0 \\
	 	0 \\
		0 \\
	 	0 \\
	\end{array}
	\right)
\end{align}
これらの連立方程式を解くと,
\begin{align}
	D_{1}^{*(I-)}=u_{x},\hspace{0.5cm}D_{2}^{*(I-)}=u_{y},\hspace{0.5cm}D_{3}^{*(I-)}=0
	\nonumber
	\\
	A_{ij}^{*(I-)}=0,\hspace{0.5cm}A_{ij}^{*(I+)}=0,\hspace{0.5cm}
	B_{ij}^{*(I-)}=0,\hspace{0.5cm}B_{ij}^{*(I+)}=0
\end{align}
以上から,$\hat{\bm{u}}^{(I)}$は以下のようになる.
\begin{align}
	\hat{u}_{r}^{(I-)}=u_{x}(\bm{x_{0}})\cos(\theta)+u_{y}(\bm{x_{0}})\sin(\theta)
	\nonumber
	\\
	\hat{u}_{\theta}^{(I-)}=-u_{x}(\bm{x_{0}})\sin(\theta)+u_{y}(\bm{x_{0}})\cos(\theta)
	\nonumber
	\\
	\hat{u}_{r}^{(I+)}=0,\hspace{0.5cm}\hat{u}_{\theta}^{(I+)}=0
\end{align}

\newpage
