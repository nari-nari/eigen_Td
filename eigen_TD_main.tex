\documentclass[titlepage,a4paper,12pt,oneside,dvipdfmx]{jsbook}
\usepackage[dvipdfmx]{graphicx}
\usepackage{amsmath,amssymb}
\usepackage{mathrsfs}
\usepackage{amsthm}
\usepackage{booktabs}
\usepackage{bm}
\usepackage{plext}
\usepackage{color} 
\usepackage{here}
%\usepackage{slashbox}
\usepackage{comment}

%\usepackage{emathUtf}
%\usepackage{subfigure}
%\usepackage[subrefformat=parens]{subcaption}
\newtheorem{theorem}{定理}
\renewcommand{\figurename}{Fig.}
\renewcommand{\tablename}{Table }
\newcommand{\dsp}{\displaystyle}
\setlength{\textheight}{20cm}
\setlength{\textwidth}{13.5cm}
%\setlength{\textheight}{22cm}
%\setlength{\textwidth}{15cm}

\setlength{\marginparwidth}{1.5cm}


	\setlength{\topmargin}{40pt}
	\iftombow
 	\addtolentgh{\topmargin}{-1in}
	\else
 	\addtolength{\topmargin}{-1truein}
	\fi

\setlength{\oddsidemargin}{0.0cm}
\setlength{\evensidemargin}{0.0cm}

\begin{document}

\chapter{形状微分}
\begin{figure}[ht]
	\begin{center}
		\includegraphics[width=13cm]{./figures/TD.png}
		\caption{Topological derivative}
		\label{fig:TD}
	\end{center}
\end{figure}

objective function
\begin{align}
\min_{\Omega_1,\Omega_2}F=-\Bigr(\sum_{p=1}^{n}\frac{1}{\lambda_p}\Bigl)^{-1}
\end{align}


状態方程式は以下のようになる。
\begin{align}
	&C_{ijkl}^{p}u_{k,lj}^{}+\lambda\rho^{p} u_{i}=0&\text{in}\hspace{0.3cm}\Omega_{p}
	\label{eq:govmain}
	\\
	&u_{i}=0 &\text{on}\hspace{0.3cm}\Gamma_{D}
	\\
	&C_{ijkl}u_{k,l}^{}n_{j}=0 &\text{on}\hspace{0.3cm}\Gamma_{N}
	\\
	&u_{i}^{p}=u_{i}^{p} &\text{on}\hspace{0.3cm}\Gamma_{pq}
	\\
	&C_{ijkl}^{p}u_{k,l}^{p}n_{j}^{p}+C_{ijkl}^{q}u_{k,l}^{q}n_{j}^{q}=0 &\text{on}\hspace{0.3cm}\Gamma_{pq}
	\label{eq:govbc}
	\\
	&\sum_{p=1}^{n}\int_{\Omega_p}(\rho^{p}u_{i}u_{i}) d\Omega=1
	\label{eq:govnorm}
\end{align}

ラグラジアンを以下のように定式化する。
\begin{align}
	\mathscr{L}(\Omega_{p\{1\leq p\leq n\}};\bm{U},X,\bm{V},Y)=
	&X+\sum_{p=1}^{n}\int_{\Omega_p}\Bigr( V_{i,j}C_{ijkl}^{p}U_{k,l}-X\rho^{p}V_{i}U_{i}\Bigl) d\Omega
	\nonumber
	\\
	&-\sum_{p=1}^{n-1}\sum_{p=q}^{n}\int_{\Gamma_{pq}}\frac{1}{2}\Bigr(C_{ijkl}^{p}U_{k,l}^{p}n_{j}^{p}-C_{ijkl}^{q}U_{k,l}^{q}n_{j}^{q}\Bigl)(V_{i}^{p}-V_{i}^{q}) d\Gamma
	\nonumber
	\\
	&-\sum_{p=1}^{n-1}\sum_{p=q}^{n}\int_{\Gamma_{pq}}\frac{1}{2}\Bigr(V_{i,j}^{p}C_{ijkl}^{p}n_{l}^{p}-V_{i,j}^{q}C_{ijkl}^{q}n_{l}^{q}\Bigl)(U_{k}^{p}-U_{k}^{q})d\Gamma
	\nonumber
	\\
	&-\sum_{p=1}^{n}\int_{\Gamma_{pu}}\Bigr(C_{ijkl}^{p}U_{k,l}^{p}n_{j}^{p}\Bigl)(V_{i}^{p}) d\Gamma
	\nonumber
	\\
	&-\sum_{p=1}^{n}\int_{\Gamma_{pu}}\Bigr(V_{i,j}^{p}C_{ijkl}^{p}n_{l}^{p}\Bigl)(U_{k}^{p}) d\Gamma
	\nonumber
	\\
	&+Y\Bigr\{1- \sum_{p=1}^{n}\int_{\Omega_{p}}(\rho^{p}U_{i}U_{i}) d\Omega \Bigl\}
	\label{eq:lagrange}
\end{align}
ここで,$\bm{U},X$はそれぞれ状態変数$\bm{u},\lambda$に対応する設計変数,$\bm{V},Y$はラグランジュ乗数である。
ただし,$\bm{u}$や$\lambda$が領域$\Omega_p$に依存する関数であるのに対し,$\bm{U},X,\bm{V},Y$は$\Omega_p$に依存しない関数である。

ラグラジアン\eqref{eq:lagrange}に対し,$\bm{V}$に関して部分積分を行うと
\begin{align}
	\mathscr{L}(\Omega_{p\{1\leq p\leq n\}};\bm{U},X,\bm{V},Y)=&X-\sum_{p=1}^{n}\int_{\Omega_p}\Bigr(C_{ijkl}^{p}U_{k,lj}+X\rho^{p}U_{i}\Bigl)V_{i} d\Omega
	\nonumber
	\\
	&+\sum_{p=1}^{n-1}\sum_{p=q}^{n}\int_{\Gamma_{pq}}\frac{1}{2}\Bigr(C_{ijkl}^{p}U_{k,l}^{p}n_{j}^{p}+C_{ijkl}^{q}U_{k,l}^{q}n_{j}^{q}\Bigl)(V_{i}^{p}-V_{i}^{q}) d\Gamma
	\nonumber
	\\
	&-\sum_{p=1}^{n-1}\sum_{p=q}^{n}\int_{\Gamma_{pq}}\frac{1}{2}\Bigr(V_{i,j}^{p}C_{ijkl}^{p}n_{l}^{p}-V_{i,j}^{q}C_{ijkl}^{q}n_{l}^{q}\Bigl)(U_{k}^{p}-U_{k}^{q})d\Gamma
	\nonumber
	\\
	&+\sum_{p=1}^{n}\int_{\Gamma_{pN}}\Bigr(C_{ijkl}^{p}U_{k,l}^{p}n_{j}^{p}\Bigl)(V_{i}^{p}) d\Gamma
	\nonumber
	\\
	&-\sum_{p=1}^{n}\int_{\Gamma_{pu}}\Bigr(V_{i,j}^{p}C_{ijkl}^{p}n_{l}^{p}\Bigl)(U_{k}^{p}) d\Gamma
	\nonumber
	\\
	&+Y\Bigr\{1- \sum_{p=1}^{n}\int_{\Omega_{p}}(\rho^{p}U_{i}U_{i}) d\Omega \Bigl\}
	\nonumber
	\\
	\label{eq:lagrange_wint}
\end{align}
状態方程式\eqref{eq:govmain}~\eqref{eq:govnorm}より,$\bm{U}=\bm{u}(\Omega_{p\{1\leq p\leq n\}})$,
$X=\lambda(\Omega_{p\{1\leq p\leq n\}})$とすれば,任意の$\bm{V},Y$に対して次式が成立することがわかる。
\begin{align}
	\mathscr{L}(\Omega_{p\{1\leq p\leq n\}};\bm{u},\lambda,\bm{U},Y)=\lambda(\Omega_{p\{1\leq p\leq n\}})
\end{align}
$\bm{V}=\bm{v}(\Omega_{p\{1\leq p\leq n\}})$,
$Y=\eta(\Omega_{p\{1\leq p\leq n\}})$


この時,$\lambda(\Omega_{p\{1\leq p\leq n\}})$の形状微分は連鎖律から次式のようになる。
\begin{align}
	D\lambda(\Omega_{p\{1\leq p\leq n\}})\cdot\bm{\theta}
	=&D\mathscr{L}(\Omega_{p\{1\leq p\leq n\}};\bm{u},\lambda,\bm{V},Y)\cdot\bm{\theta}
	\nonumber
	\\
	&+\Bigr\langle \frac{\partial \mathscr{L}}{\partial \bm{V}}(\Omega_{p\{1\leq p\leq n\}};\bm{u},\lambda,\bm{V},Y),
	\bm{u}'(\Omega_{p\{1\leq p\leq n\}}) \Bigl\rangle
	\nonumber
	\\
	&+\frac{\partial \mathscr{L}}{\partial X}(\Omega_{p\{1\leq p\leq n\}};\bm{u},\lambda,\bm{V},Y)\cdot
	\lambda'(\Omega_{p\{1\leq p\leq n\}})
	\label{eq:shape_lagrange}
\end{align}
ここで,$\bm{u}$や$\lambda$の変分を計算しなくてよい条件を求めるために,ラグラジアンの停留条件を考える。
\begin{align}
	\left.\bigr\langle \frac{\partial \mathscr{L}}{\partial \bm{U}},\delta \bm{U} \bigl\rangle\right|_{opt}=0
	\label{eq:optc_v}
	\\
	\left.\frac{\partial\mathscr{L}}{\partial X}\right|_{opt}=0
	\label{eq:optc_x}
	\\
	\left.\bigr\langle \frac{\partial \mathscr{L}}{\partial \bm{V}},\delta \bm{V} \bigl\rangle\right|_{opt}=0
	\label{eq:optc_w}
	\\
	\left.\frac{\partial\mathscr{L}}{\partial Y}\right|_{opt}=0
	\label{eq:optc_y}
\end{align}
まず,式\eqref{eq:optc_w},\eqref{eq:optc_y}に関しては,式\eqref{eq:lagrange_wint}より次式のようになる。
\begin{align}
	0=&\left.\bigr\langle \frac{\partial \mathscr{L}}{\partial \bm{V}},\delta \bm{V} \bigl\rangle\right|_{opt}
	\nonumber
	\\
	=&\sum_{p=1}^{n}\int_{\Omega_p}\Bigr(C_{ijkl}^{p}U_{k,lj}+X\rho^{p}U_{i}\Bigl)\delta V_{i} d\Omega
	\nonumber
	\\
	&+\sum_{p=1}^{n-1}\sum_{p=q}^{n}\int_{\Gamma_{pq}}\frac{1}{2}\Bigr(C_{ijkl}^{p}U_{k,l}^{p}n_{j}^{p}+C_{ijkl}^{q}U_{k,l}^{q}n_{j}^{q}\Bigl)(\delta V_{i}^{p}-\delta V_{i}^{q}) d\Gamma
	\nonumber
	\\
	&-\sum_{p=1}^{n-1}\sum_{p=q}^{n}\int_{\Gamma_{pq}}\frac{1}{2}\Bigr(\delta V_{i,j}^{p}C_{ijkl}^{p}n_{l}^{p}-\delta V_{i,j}^{q}C_{ijkl}^{q}n_{l}^{q}\Bigl)(U_{k}^{p}-U_{k}^{q})d\Gamma
	\nonumber
	\\
	&+\sum_{p=1}^{n}\int_{\Gamma_{pN}}\Bigr(U_{i,j}^{p}C_{ijkl}^{p}n_{l}^{p}\Bigl)(\delta V_{i}^{p}) d\Gamma
	\nonumber
	\\
	&-\sum_{p=1}^{n}\int_{\Gamma_{pu}}\Bigr(\delta V_{i,j}^{p}C_{ijkl}^{p}n_{l}^{p}\Bigl)(U_{k}^{p}) d\Gamma
	\label{eq:lagrange_wderivative}
\end{align}
\begin{align}
	0=\left.\frac{\partial \mathscr{L}}{\partial Y}\right|_{opt}=1- \sum_{p=1}^{n}\int_{\Omega_{p}}(\rho^{p}U_{i}U_{i}) d\Omega
	\label{eq:lagrange_yderivative}
\end{align}
上式と状態方程式\eqref{eq:govmain}~\eqref{eq:govnorm}との比較から,$\bm{U}=\bm{u}(\Omega_{p\{1\leq p\leq n\}})$,$X=\lambda(\Omega_{p\{1\leq p\leq n\}})$の時,
任意の$\delta \bm{V}$に対して停留条件\eqref{eq:optc_w},\eqref{eq:optc_y}が成立することがわかる。
続いて,$\bm{U},X$に関する停留条件\eqref{eq:optc_v},\eqref{eq:optc_x}をみたすような,
$\bm{V}=\bm{v}(\Omega_{p\{1\leq p\leq n\}}),Y=\eta(\Omega_{p\{1\leq p\leq n\}})$について考える.

ラグラジアン\eqref{eq:lagrange}に対し,$\bm{U}$に関して部分積分を行うと
\begin{align}
	\mathscr{L}(\Omega_{p\{1\leq p\leq n\}};\bm{U},X,\bm{V},Y)=&X-\sum_{p=1}^{n}\int_{\Omega_p}\Bigr(V_{i,jl}C_{ijkl}^{p}+X\rho^{p}V_{k}\Bigl)U_{k} d\Omega
	\nonumber
	\\
	&-\sum_{p=1}^{n-1}\sum_{p=q}^{n}\int_{\Gamma_{pq}}\frac{1}{2}\Bigr(C_{ijkl}^{p}U_{k,l}^{p}n_{j}^{p}-C_{ijkl}^{q}U_{k,l}^{q}n_{j}^{q}\Bigl)(V_{i}^{p}-V_{i}^{q}) d\Gamma
	\nonumber
	\\
	&+\sum_{p=1}^{n-1}\sum_{p=q}^{n}\int_{\Gamma_{pq}}\frac{1}{2}\Bigr(V_{i,j}^{p}C_{ijkl}^{p}n_{l}^{p}+V_{i,j}^{q}C_{ijkl}^{q}n_{l}^{q}\Bigl)(U_{k}^{p}-U_{k}^{q})d\Gamma
	\nonumber
	\\
	&+\sum_{p=1}^{n}\int_{\Gamma_{pN}}\Bigr(V_{i,j}^{p}C_{ijkl}^{p}n_{l}^{p}\Bigl)(U_{k}^{p}) d\Gamma
	\nonumber
	\\
	&-\sum_{p=1}^{n}\int_{\Gamma_{pu}}\Bigr(C_{ijkl}^{p}U_{k,l}^{p}n_{j}^{p}\Bigl)(V_{i}^{p}) d\Gamma
	\nonumber
	\\
	&+Y\Bigr\{1- \sum_{p=1}^{n}\int_{\Omega_{p}}(\rho^{p}U_{i}U_{i}) d\Omega \Bigl\}
	\nonumber
	\\
	\label{eq:lagrange_vint}
\end{align}
となるため,停留点における$\bm{V}$および$Y$をそれぞれ$\bm{v}$,$\eta$とすると,式\eqref{eq:optc_v},\eqref{eq:optc_x}は以下のようになる。
\begin{align}
	0=&\left.\bigr\langle \frac{\partial \mathscr{L}}{\partial \bm{U}},\delta \bm{U} \bigl\rangle\right|_{opt}
	\nonumber
	\\
	=&\sum_{p=1}^{n}\int_{\Omega_p}\Bigr(v_{k,lj}C_{ijkl}^{p}+\lambda\rho^{p}v_{i}+2\eta\rho^{p}u_{i}\Bigl)\delta U_{i} d\Omega
	\nonumber
	\\
	&-\sum_{p=1}^{n-1}\sum_{p=q}^{n}\int_{\Gamma_{pq}}\frac{1}{2}\Bigr(\delta U_{i,j}^{p}C_{ijkl}^{p}n_{l}^{p}-\delta U_{i,j}^{q}C_{ijkl}^{q}n_{l}^{q}\Bigl)(v_{k}^{p}-v_{k}^{q}) d\Gamma
	\nonumber
	\\
	&+\sum_{p=1}^{n-1}\sum_{p=q}^{n}\int_{\Gamma_{pq}}\frac{1}{2}\Bigr(v_{k,l}^{p}C_{ijkl}^{p}n_{j}^{p}+v_{k,l}^{q}C_{ijkl}^{q}n_{j}^{q}\Bigl)(\delta U_{i}^{p}-\delta U_{i}^{q})d\Gamma
	\nonumber
	\\
	&+\sum_{p=1}^{n}\int_{\Gamma_{pN}}\Bigr(v_{k,l}^{p}C_{ijkl}^{p}n_{j}^{p}\Bigl)\delta U_{i}^{p} d\Gamma
	\nonumber
	\\
	&-\sum_{p=1}^{n}\int_{\Gamma_{pu}}\Bigr(\delta U_{i,j}^{p}C_{ijkl}^{p}n_{l}^{p}\Bigl)v_{k}^{p} d\Gamma
	\label{eq:lagrange_vderivative}
\end{align}
\begin{align}
	0=\left.\frac{\partial\mathscr{L}}{\partial X}\right|_{opt}
	=1-\sum_{p=1}^{n}\int_{\Omega_p}\bigr(\rho^{p}v_{i}u_{i}\bigl) d\Omega
	\label{eq:lagrange_xderivative}
\end{align}
ここで,$C_{ijkl}=C_{klij}$であることを用いて,添え字を$(i,j,k,l)\rightarrow(k,l,i,j)$に変換した.
以上から,$\bm{v}$,$\eta$が以下の随伴方程式を満たすとき,停留条件\eqref{eq:optc_v},\eqref{eq:optc_x}は成立する。
\begin{align}
	&C_{ijkl}^{p}v_{k,lj}+\lambda\rho^{p}v_{i}+2\eta\rho^{p} u_{i}=0&\text{in}\hspace{0.3cm}\Omega_{p}
	\label{eq:adjmain}
	\\
	&v_{i}=0 &\text{on}\hspace{0.3cm}\Gamma_{D}
	\\
	&C_{ijkl}v_{k,l}^{}n_{j}=0 &\text{on}\hspace{0.3cm}\Gamma_{N}
	\\
	&v_{i}^{p}=v_{i}^{q} &\text{on}\hspace{0.3cm}\Gamma_{pq}
	\\
	&C_{ijkl}^{p}v_{k,l}^{p}n_{j}^{p}+C_{ijkl}^{q}v_{k,l}^{q}n_{j}^{q}=0 &\text{on}\hspace{0.3cm}\Gamma_{pq}
	\label{eq:adjbc}
	\\
	&\sum_{p=1}^{n}\int_{\Omega_p}(\rho^{p}v_{i}u_{i}) d\Omega=1
	\label{eq:adjnorm}
\end{align}
次に,状態場の正則制約のラグランジュ乗数である$\eta$の停留点における値を求める。
まず,強形式の支配方程式\eqref{eq:govmain}に$v_i$をかけ,積分すると次式が得られる。
\begin{align}
	\sum_{p=1}^{n}\int_{\Omega_p}\Bigr(v_{i,j}C_{ijkl}^{p}u_{k,l}-\lambda\rho^{p}v_{i}u_{i}\Bigl) d\Omega=0
	\label{eq:gov_p}
\end{align}
強形式の随伴方程式\eqref{eq:adjmain}に$u_i$をかけ,積分すると次式が得られる。
\begin{align}
	\sum_{p=1}^{n}\int_{\Omega_p}\Bigr(v_{i,j}C_{ijkl}^{p}u_{k,l}-\lambda\rho^{p}u_{i}^{}v_{i}^{}-2\eta u_{i}^{}u_{i}^{}\Bigl) d\Omega=0
	\label{eq:adj_u}
\end{align}
式\eqref{eq:gov_p},\eqref{eq:adj_u},\eqref{eq:govnorm}より,
\begin{align}
	\eta\sum_{p=1}^{n}\int_{\Omega_p}(\rho^{p}u_{i}^{}u_{i}^{}) d\Omega=\eta=0
	\label{eq:yopt}
\end{align}
ゆえに,随伴方程式\eqref{eq:adjmain}~\eqref{eq:adjbc}は状態方程式\eqref{eq:govmain}~\eqref{eq:govbc}と同じ形になり,
随伴場$\bm{v}$は次式のように状態場$\bm{u}$の定数倍になる。
\begin{align}
	\bm{v}=K\bm{u}
	\label{eq:v_ku}
\end{align}
ここで,式\eqref{eq:govnorm},式\eqref{eq:adjnorm}より,
\begin{align}
	1&=\sum_{p=1}^{n}\int_{\Omega_p}\Bigr(\rho^{p}u_{i}v_{i}\Bigl) d\Omega
	\nonumber
	\\
	&=K\sum_{p=1}^{n}\int_{\Omega_p}\Bigr(\rho^{p}u_{i}u_{i}\Bigl) d\Omega
	\nonumber
	\\
	&=K
	\label{eq:k}
\end{align}
ゆえに,
\begin{align}
	\bm{v}=\bm{u}
	\label{eq:v_u}
\end{align}
以上から,$\lambda(\Omega_{p\{1\leq p\leq n\}})$の形状微分は次式のようになる。
\begin{align}
	D\lambda(\Omega_{p\{1\leq p\leq n\}})\cdot\bm{\theta}
	=&D\mathscr{L}(\Omega_{p\{1\leq p\leq n\}};\bm{u},\lambda,\bm{v},\eta)\cdot\bm{\theta}
	\nonumber
	\\
	&+\Bigr\langle \frac{\partial \mathscr{L}}{\partial \bm{U}}(\Omega_{p\{1\leq p\leq n\}};\bm{u},\lambda,\bm{v},\eta),
	\bm{u}'(\Omega_{p\{1\leq p\leq n\}}) \Bigl\rangle
	\nonumber
	\\
	&+\frac{\partial \mathscr{L}}{\partial X}(\Omega_{p\{1\leq p\leq n\}};\bm{p},\lambda,\bm{v},\eta)\cdot
	\lambda'(\Omega_{p\{1\leq p\leq n\}})
	\nonumber
	\\
	&+\Bigr\langle \frac{\partial \mathscr{L}}{\partial \bm{V}}(\Omega_{p\{1\leq p\leq n\}};\bm{u},\lambda,\bm{v},\eta),
	\bm{v}'(\Omega_{p\{1\leq p\leq n\}}) \Bigl\rangle
	\nonumber
	\\
	&+\frac{\partial \mathscr{L}}{\partial Y}(\Omega_{p\{1\leq p\leq n\}};\bm{p},\lambda,\bm{v},\eta)\cdot
	\eta'(\Omega_{p\{1\leq p\leq n\}})
	\nonumber
	\\
	=&D\mathscr{L}(\Omega_{p\{1\leq p\leq n\}};\bm{u},\lambda,\bm{u},0)\cdot\bm{\theta}
	\label{eq:shape_lambda_u}
\end{align}

ここで,形状微分の公式を引用する。汎関数$J$が積分領域$\Omega$に依存しない密度関数$f$の領域積分によって,次式のように表されるとする。
\begin{align}
	J=\int_{\Omega}f d\Omega
	\label{eq:functinal_d}
\end{align}
この時,$J$の形状微分は次式のようになる。
\begin{align}
	DJ\cdot\theta=\int_{\Gamma}(\theta_{i}n_{i})f d\Omega
	\label{eq:drivative_d}
\end{align}
ただし,$\bm{n}$は$\Gamma$上の外向きの単位法線ベクトルを表す。
一方で,汎関数$J$が密度関数$f$の境界積分によって,次式のように表されるとする。
\begin{align}
	J=\int_{\Gamma}f d\Gamma
	\label{eq:functinal_b}
\end{align}
この時,$J$の形状微分は次式のようになる。
\begin{align}
	DJ\cdot\theta=\int_{\Gamma}(\theta_{i}n_{i})\Bigr(\frac{\partial f}{\partial x_{j}}n_{j}+Hf\Bigl) d\Omega
	\label{eq:drivative_b}
\end{align}
ただし,$H\equiv div\bm{n}$は$\Gamma$の平均曲率を表す。
式\eqref{eq:drivative_d},\eqref{eq:drivative_b}を式\eqref{eq:shape_lagrange}のラグラジアンに用い,
$\Gamma_{u}$上で$\bm{\theta}=\bm{0}$を仮定すると,$\lambda(\Omega_{p\{1\leq p\leq n\}})$の形状微分は次式のようになる。
\begin{align}
	&\hspace{0.5cm}D\lambda(\Omega_{p\{1\leq p\leq n\}})\cdot\bm{\theta}
	\nonumber
	\\
	&=D\mathscr{L}(\Omega_{p\{1\leq p\leq n\}};\bm{u},\lambda,\bm{u},0)\cdot\bm{\theta}
	\nonumber
	\\
	&=\sum_{p=1}^{n}\int_{\Gamma_p}(\theta_{m}n_{m}^{p})\Bigr(u_{i,j}C_{ijkl}^{p}u_{k,l}-\lambda\rho^{p}u_{i}u_{i}\Bigl) d\Omega
	\nonumber
	\\
	&-\sum_{p=1}^{n-1}\sum_{p=q}^{n}\int_{\Gamma_{pq}}(\theta_{m}n_{m}^{p})
	\Bigr(\frac{\partial}{\partial x_\gamma }n_\gamma^{p} +H^{p}\Bigl)
	\Bigr(C_{ijkl}^{p}u_{k,l}^{p}n_{j}^{p}-C_{ijkl}^{q}u_{k,l}^{q}n_{j}^{q}\Bigl)
	(u_{i}^{p}-u_{i}^{q}) d\Gamma
	\label{eq:shape_lambda}
\end{align}
境界条件から$\partial \Omega_{pq}$上で$\bm{u}^{p}=\bm{u}^{q}$であることに注意すると以下のように整理される。
\begin{align}
	&\hspace{0.5cm}D\lambda(\Omega_{p\{1\leq p\leq n\}})\cdot\bm{\theta}
	\nonumber
	\\
	&=\sum_{p=1}^{n}\int_{\Gamma_p}(\theta_{m}n_{m}^{p})\Bigr(u_{i,j}C_{ijkl}^{p}u_{k,l}-\lambda\rho^{p}u_{i}u_{i}\Bigl) d\Omega
	\nonumber
	\\
	&-\sum_{p=1}^{n-1}\sum_{p=q}^{n}\int_{\Gamma_{pq}}(\theta_{m}n_{m}^{p})
	\Bigr(C_{ijkl}^{p}u_{k,l}^{p}n_{j}^{p}-C_{ijkl}^{q}u_{k,l}^{q}n_{j}^{q}\Bigl)
	(u_{i,\gamma}^{p}n_\gamma^{p}-u_{i,\gamma}^{q}n_\gamma^{p}) d\Gamma
	\label{eq:shape_lambda}
\end{align}


\chapter{固有振動モードの漸近展開}
\section{固有振動モードの変動に関する方程式}

\begin{figure}[ht]
	\begin{center}
		\includegraphics[width=13cm]{./figures/TD.png}
		\caption{Topological derivative}
		\label{fig:TD}
	\end{center}
\end{figure}

材料$a$で占められた領域$\Omega_a$中の微小な円領域$\Omega_\epsilon$に材料$b$が生成した際の,固有振動モード$\bm{u}^{\epsilon}$および固有値$\lambda^{\epsilon}$について考える.
ただし,これによって固有振動モードの次数の変化は起こらないと仮定する.
この時の支配方程式は以下のようになる.
\begin{align}
	&C_{ijkl}^{b}u_{k,lj}^{\epsilon}+\lambda^{\epsilon}\rho^{b} u_{i}^{\epsilon}=0&&\text{in}\hspace{0.3cm}\Omega_{\epsilon}
	\label{eq:govepmain}
	\\
	&C_{ijkl}^{a}u_{k,lj}^{\epsilon}+\lambda^{\epsilon}\rho^{a} u_{i}^{\epsilon}=0&&\text{in}\hspace{0.3cm}\Omega_{a}\backslash\Omega_{\epsilon}
	\label{eq:govepmain}
	\\
	&C_{ijkl}^{p}u_{k,lj}^{\epsilon}+\lambda^{\epsilon}\rho^{p} u_{i}^{\epsilon}=0&&\text{in}\hspace{0.3cm}\Omega_{p}\hspace{0.3cm}(p\neq a)
	\label{eq:govepmain}
	\\
	&u_{i}^{\epsilon b}=u_{i}^{\epsilon a} &&\text{on}\hspace{0.3cm}\Gamma_{\epsilon}
	\\
	&C_{ijkl}^{b}u_{k,l}^{\epsilon b}n_{j}^{b}+C_{ijkl}^{a}u_{k,l}^{\epsilon a}n_{j}^{a}=0 &&\text{on}\hspace{0.3cm}\Gamma_{\epsilon}
	\label{eq:govepbc}
	\\
	&u_{i}^{\epsilon p}=u_{i}^{\epsilon q} &&\text{on}\hspace{0.3cm}\Gamma_{pq}
	\\
	&C_{ijkl}^{p}u_{k,l}^{\epsilon p}n_{j}^{p}+C_{ijkl}^{q}u_{k,l}^{\epsilon q}n_{j}^{q}=0 &&\text{on}\hspace{0.3cm}\Gamma_{pq}
	\label{eq:govepbc}
	\\
	&\sum_{p=1,p\neq a}^{n}\int_{\Omega_p}(\rho^{p}u_{i}^{\epsilon}u_{i}^{\epsilon}) d\Omega
	+\int_{\Omega_a\backslash\Omega_{\epsilon}}(\rho^{a}u_{i}^{\epsilon}u_{i}^{\epsilon}) d\Omega
	+\int_{\Omega_{\epsilon}}(\rho^{b}u_{i}^{\epsilon}u_{i}^{\epsilon}) d\Omega=1
	\label{eq:govepnorm}
\end{align}

$\bm{u}^{\epsilon},\lambda^{\epsilon}$を,微小な材料$b$の領域が生成される前の固有ベクトルおよび固有値$\bm{u},\lambda$用いて展開すると,次式のようになる。
\begin{align}
	\bm{u}^{\epsilon}(\bm{x})=&\bm{u}(\bm{x})\chi_{R^2/\Omega}+\hat{\bm{u}}(\bm{x})
	\\
	\lambda^{\epsilon}=&\lambda+\hat{\lambda}
\end{align}
ここで,$\Omega_\epsilon$の中心座標を$\bm{x}_0$とし,$\bm{\xi}=\frac{\bm{x}-\bm{x}_0}{\epsilon}$を導入し,
$\hat{\bm{u}}(\bm{x})$を$\epsilon$に関して次式のように漸近展開する.
\begin{align}
	\hat{\bm{u}}(\bm{x})
	=&\{\hat{\bm{u}}^{(I-)}(\bm{x})+\epsilon\hat{\bm{u}}^{(I\hspace{-.1em}I-)}(\bm{x})\}\chi_{\Omega_\epsilon}
	+\{\hat{\bm{u}}^{(I+)}(\bm{x})+\epsilon\hat{\bm{u}}^{(I\hspace{-.2em}I+)}(\bm{x})\}\chi_{R^2\backslash\Omega_\epsilon}+O(\epsilon^2)
	\nonumber
	\\
	=&\{\hat{\bm{w}}^{(I-)}(\bm{\xi})+\epsilon\hat{\bm{w}}^{(I\hspace{-.1em}I-)}(\bm{\xi})\}\chi_{\Omega_\epsilon}
	+\{\hat{\bm{w}}^{(I+)}(\bm{\xi})+\epsilon\hat{\bm{w}}^{(I\hspace{-.2em}I+)}(\bm{\xi})\}\chi_{R^2\backslash\Omega_\epsilon}+O(\epsilon^2)
	\nonumber
	\\
	=&\{\hat{\bm{w}}^{(-)}(\bm{\xi})\}\chi_{\Omega_\epsilon}
	+\{\hat{\bm{w}}^{(+)}(\bm{\xi})\}\chi_{R^2\backslash\Omega_\epsilon}+O(\epsilon^2)
\end{align}
ただし,上付き添え字の+は微小領域の外側を-は内側を示している.
この時,$\hat{\bm{w}}(\bm{\xi})^{(\pm)}$は次式を満たす.
\begin{align}
	&C_{ijkl}^{b} \Bigl \{ \frac{1}{\epsilon^2}\frac{\partial^2 \hat{w}^{(-)}_{k}} {\partial \xi_l \partial \xi_j}(\bm{\xi})\Bigr \}
	+\bigl( \lambda+\hat{\lambda}\bigr) \rho^{b} 
	 \hat{w}_{i}^{(-)}(\bm{\xi})=0
	&&\text{in}\hspace{0.3cm}\Omega_{\epsilon}
	\label{eq:GovEpsIn}
	\\
	&C_{ijkl}^{a} \Bigl \{ \frac{\partial^2 u^{}_{k}} {\partial x_l \partial x_j}(\bm{x})
	+\frac{1}{\epsilon^2}\frac{\partial^2 \hat{w}^{(+)}_{k}} {\partial \xi_l \partial \xi_j}(\bm{\xi})\Bigr \}
	+\bigl( \lambda+\hat{\lambda}\bigr) \rho^{a}
	\bigl \{ u_{i}^{}(\bm{x})+\hat{w}_{i}^{(+)}(\bm{\xi})\bigr \}=0
	&&\text{in}\hspace{0.3cm}\Omega_{a}\backslash\Omega_{\epsilon}
	\label{eq:GovEpsOut}
	\\
	&\hat{w}_{i}^{(-)}(\bm{\xi})=u_{i}^{}(\bm{x})+\hat{w}_{i}^{(+)}(\bm{\xi})
	&&\text{on}\hspace{0.3cm}\Gamma_{\epsilon}
	\\
	&C_{ijkl}^{b} \Bigl \{ \frac{1}{\epsilon}\frac{\partial \hat{w}^{(-)}_{k}} {\partial \xi_l}(\bm{\xi}) n_{j}^{(-)} \Bigr \}
	+C_{ijkl}^{a} \Bigl \{ \frac{\partial u^{}_{k}} {\partial x_l}(\bm{x})n_{j}^{(+)}
	+\frac{1}{\epsilon}\frac{\partial \hat{w}^{(+)}_{k}} {\partial \xi_l}(\bm{\xi}) n_{j}^{(+)} \Bigr \}
	=0
	&&\text{on}\hspace{0.3cm}\Gamma_{\epsilon}
	\label{eq:GovEpsBC}
\end{align}
\begin{align}
	&\sum_{p=1,p\neq a}^{n}\int_{\Omega_p}\rho^{p}(u_{i}u_{i}+2u_{i}\hat{w}_{i}^{(+)}+\hat{w}_{i}^{(+)}\hat{w}_{i}^{(+)}) d\Omega
	\nonumber
	\\
	&\hspace{1.0cm}+\int_{\Omega_a\backslash\Omega_{\epsilon}}\rho^{a}(u_{i}u_{i}+2u_{i}\hat{w}_{i}^{(+)}+\hat{w}_{i}^{(+)}\hat{w}_{i}^{(+)}) d\Omega
	+\int_{\Omega_{\epsilon}}\rho^{b}(\hat{w}_{i}^{(-)}\hat{w}_{i}^{(-)}) d\Omega=1
	\label{eq:GovEpsNorm}
\end{align}

式\eqref{eq:GovEpsOut}から式\eqref{eq:Gov}を引き,整理することで,以下のようになる.
\begin{align}
	&C_{ijkl}^{b} \hat{w}^{(-)}_{k,lj}(\bm{\xi})
	+\epsilon^2\Bigl\{(\lambda+\hat{\lambda})\rho^{b} \hat{w}_{i}^{(-)}(\bm{\xi})\Bigr\}=0
	&&\text{in}\hspace{0.3cm}\Omega_{\epsilon}
	\label{eq:GovDisturIn}
	\\
	&C_{ijkl}^{a}\hat{w}^{(+)}_{k,lj}(\bm{\xi})
	+\epsilon^2\Bigl \{\lambda\rho^{a} \hat{w}_{i}^{(+)}(\bm{\xi})
	+\hat{\lambda}\rho^{a} \Bigl(u_{i}^{}(\bm{x})+\hat{w}_{i}^{(+)}(\bm{\xi}) \Bigr) \Bigr \}=0
	&&\text{in}\hspace{0.3cm}\Omega_{a}\backslash\Omega_{\epsilon}
	\label{eq:GovDisturOut}
	\\
	&\hat{w}_{i}^{(-)}(\bm{\xi})-\hat{w}_{i}^{(+)}(\bm{\xi})=u_{i}(\bm{x})=u_{i}(\bm{x_0})+\epsilon \xi_j u_{i,j}(\bm{x_0})+O(\epsilon^2)
	&&\text{on}\hspace{0.3cm}\Gamma_{\epsilon}
	\label{eq:GovDisturUBC}
	\\
	&C_{ijkl}^{b}\hat{w}_{k,l}^{(-)}(\bm{\xi})n_{j}^{(-)}
	-C_{ijkl}^{a}\hat{w}_{k,l}^{(+)}(\bm{\xi})n_{j}^{(-)}
	=\epsilon C_{ijkl}^{a}u_{k,l}(\bm{x_0})n_{j}^{(-)}+O(\epsilon^2)
	&&\text{on}\hspace{0.3cm}\Gamma_{\epsilon}
	\label{eq:GovDisturSBC}
\end{align}
\begin{align}
	\sum_{p=1}^{n}\int_{\Omega_p}\rho^{p}(2u_{i}\hat{w}_{i}^{(+)}+\hat{w}_{i}^{(+)}\hat{w}_{i}^{(+)}) d\Omega
	+\int_{\Omega_{\epsilon}}\{\rho^{b}\hat{w}_{i}^{(-)}\hat{w}_{i}^{(-)}-\rho^{a}u_{i}u_{i}\} d\Omega=0
	\label{eq:GovDisturNorm}
%	\\
%	\hat{w}_{i}^{(+)}(\bm{\xi})\rightarrow 0 \hspace{0.3cm} (|\bm{\xi}|\rightarrow \infty)
%	\hspace{4.5cm}
%	\label{eq:GovDisturInftyBC}
\end{align}

$\epsilon$の0次の項を整理すると
\begin{align}
	C_{ijkl}^{b} \hat{w}^{(I-)}_{k,lj}(\bm{\xi})=0
	\hspace{1.7cm}\text{in}\hspace{0.3cm}\Omega_{\epsilon}
	\label{eq:GovDisturIn1}
	\\
	C_{ijkl}^{a} \hat{w}^{(I+)}_{k,lj}(\bm{\xi})=0
	\hspace{1.2cm}\text{in}\hspace{0.3cm}\Omega\backslash\Omega_{\epsilon}
	\label{eq:GovDisturOut1}
	\\
	\hat{w}_{i}^{(I-)}(\bm{\xi})-\hat{w}_{i}^{(I+)}(\bm{\xi})=u_{i}(\bm{x_0})
	\hspace{0.7cm}\text{on}\hspace{0.3cm}\Gamma_{\epsilon}
	\label{eq:GovDisturUBC1}
	\\
	C_{ijkl}^{b}\hat{w}_{k,l}^{(I-)}(\bm{\xi})n_{j}^{(-)}
	-C_{ijkl}^{a}\hat{w}_{k,l}^{(I+)}(\bm{\xi})n_{j}^{(-)}=0
	\hspace{1.5cm}\text{on}\hspace{0.3cm}\Gamma_{\epsilon}
	\label{eq:GovDisturSBC1}
%	\\
%	\hat{w}_{i}^{(I+)}(\bm{\xi})\rightarrow 0\hspace{0.3cm} (|\bm{\xi}|\rightarrow \infty)
%	\hspace{2.0cm}
%	\label{eq:GovDisturInftyBC1}
\end{align}

$\epsilon$の1次の項を整理すると
\begin{align}
	C_{ijkl}^{b} \hat{w}^{(I\hspace{-.15em}I-)}_{k,lj}(\bm{\xi})=0
	\hspace{2.5cm}
	\text{in}\hspace{0.3cm}\Omega_{\epsilon}
	\label{eq:GovDisturIn2}
	\\
	C_{ijkl}^{a} \hat{w}^{(I\hspace{-.15em}I+)}_{k,lj}(\bm{\xi})=0
	\hspace{2.0cm}
	\text{in}\hspace{0.3cm}\Omega\backslash\Omega_{\epsilon}
	\label{eq:GovDisturOut2}
	\\
	\hat{w}_{i}^{(I\hspace{-.15em}I-)}(\bm{\xi})
	-\hat{w}_{i}^{(I\hspace{-.15em}I+)}(\bm{\xi})= \xi_j u_{i,j}(\bm{x_0})
	\hspace{2.0cm}
	\nonumber
	\\
	\equiv u_{i}^{(I\hspace{-.15em}I)}
	\hspace{1.9cm}\text{on}\hspace{0.3cm}\Gamma_{\epsilon}
	\label{eq:GovDisturUBC2}
	\\
	C_{ijkl}^{b}\hat{w}_{k,l}^{(I\hspace{-.15em}I-)}(\bm{\xi})n_{j}^{(-)}
	-C_{ijkl}^{a}\hat{w}_{k,l}^{(I\hspace{-.15em}I+)}(\bm{\xi})n_{j}^{(-)}
	=C_{ijkl}^{a} u_{k,l}^{}(\bm{x_0})n_{j}^{(-)}
	\hspace{1.0cm}
	\nonumber
	\\
	\equiv t_{i}^{(I\hspace{-.15em}I)}
	\hspace{2.0cm}\text{on}\hspace{0.3cm}\Gamma_{\epsilon}
	\label{eq:GovDisturSBC2}
%	\\
%	\hat{w}_{i}^{(I\hspace{-.15em}I+)}(\bm{\xi})\rightarrow 0 \hspace{0.3cm} (|\bm{\xi}|\rightarrow \infty)
%	\hspace{3.0cm}
%	\label{eq:GovDisturInftyBC2}
\end{align}
以上から,$\hat{\bm{w}}^{(I-)}$,$\hat{\bm{w}}^{(I+)}$,$\hat{\bm{w}}^{(I\hspace{-.15em}I-)}$,$\hat{\bm{w}}^{(I\hspace{-.15em}I+)}$は,それぞれ平衡方程式に従うことが分かる.

\newpage

\include{Asymp_S2_GeneralSol}
\include{Asymp_S3_1Inner}
\include{Asymp_S4_1Outer}
\include{Asymp_S5_1BC}
\include{Asymp_S6_1SimulEq}
\include{Asymp_S7_2BC}
\include{Asymp_S7_2_2SimulEq}

\chapter{トポロジー導関数}
\section{$\theta$の設定}

\begin{figure}[ht]
	\begin{center}
		\includegraphics[width=13cm]{./figures/SDforTD.png}
		\caption{Topological derivative}
		\label{fig:TD}
	\end{center}
\end{figure}

$\bm{\theta}$を次式のように設定する
\begin{align}
	\bm{\theta}=\theta_{n}\bm{n}^{(-)}\hspace{1cm}\text{on}\hspace{0.3cm}\Gamma_{\epsilon}
\end{align}
この時,固有値$\lambda$の形状微分は,\eqref{eq:shape_lambda}に
$\bm{u}=\bm{u}^{\epsilon},\lambda=\lambda^{\epsilon}$を代入することで以下のようになる.
\begin{align}
	&\hspace{0.5cm}D\lambda(\Omega_{p\{1\leq p\leq n\}})\cdot\bm{\theta}
	\nonumber
	\\
	=&\int_{\Gamma_\epsilon}(\theta_{n}n_{m}^{(-)}n_{m}^{(-)})
	\Bigr(u_{i,j}^{\epsilon(-)}C_{ijkl}^{b}u_{k,l}^{\epsilon(-)}
	-\lambda^{\epsilon}\rho^{b}u_{i}^{\epsilon(-)}u_{i}^{\epsilon(-)}\Bigl) d\Omega
	\nonumber
	\\
	&+\int_{\Gamma_\epsilon}(\theta_{n}n_{m}^{(-)}n_{m}^{(+)})
	\Bigr(u_{i,j}^{\epsilon(+)}C_{ijkl}^{a}u_{k,l}^{\epsilon(+)}
	-\lambda^{\epsilon}\rho^{a}u_{i}^{\epsilon(+)}u_{i}^{\epsilon(+)}\Bigl) d\Omega
	\nonumber
	\\
	&-\int_{\Gamma_{\epsilon}}(\theta_{n}n_{m}^{(-)}n_{m}^{(-)})
	\Bigr(C_{ijkl}^{b}u_{k,l}^{\epsilon(-)}n_{j}^{(-)}
	-C_{ijkl}^{a}u_{k,l}^{\epsilon(+)}n_{j}^{(+)}\Bigl)
	(u_{i,\gamma}^{\epsilon(-)}n_\gamma^{(-)}-u_{i,\gamma}^{\epsilon(+)}n_\gamma^{(-)}) d\Gamma
	\nonumber
	\\
	=&\theta_{n}\int_{\Gamma_\epsilon}
	\Bigr(u_{i,j}^{\epsilon(-)}C_{ijkl}^{b}u_{k,l}^{\epsilon(-)}
	-\lambda^{\epsilon}\rho^{b}u_{i}^{\epsilon(-)}u_{i}^{\epsilon(-)}\Bigl) d\Omega
	\nonumber
	\\
	&-\theta_{n}\int_{\Gamma_\epsilon}
	\Bigr(u_{i,j}^{\epsilon(+)}C_{ijkl}^{a}u_{k,l}^{\epsilon(+)}
	-\lambda^{\epsilon}\rho^{a}u_{i}^{\epsilon(+)}u_{i}^{\epsilon(+)}\Bigl) d\Omega
	\nonumber
	\\
	&-\theta_{n}\int_{\Gamma_{\epsilon}}
	\Bigr(C_{ijkl}^{b}u_{k,l}^{\epsilon(-)}n_{j}^{(-)}
	+C_{ijkl}^{a}u_{k,l}^{\epsilon(+)}n_{j}^{(-)}\Bigl)
	(u_{i,\gamma}^{\epsilon(-)}n_\gamma^{(-)}-u_{i,\gamma}^{\epsilon(+)}n_\gamma^{(-)}) d\Gamma
	\label{eq:SDForTD}
\end{align}
ここで,
\begin{align}
	x_{1}-x_{1_{0}}=R\cos(\theta)
	\nonumber
	\\
	x_{2}-x_{2_{0}}=R\sin(\theta)
	\label{eq:xyToRTh}
\end{align}
と座標変換すると$R,\theta$方向の変位は以下のようになる.
\begin{align}
	u_{R}^{\epsilon(-)}(R,\theta)=&u_{x_{1}}(\bm{x}_{0})\cos(\theta)+u_{x_{2}}(\bm{x}_{0})\sin(\theta)
	+\epsilon\hat{w}_{r}^{(I\hspace{-.15em}I-)}(r,\theta)+O(\epsilon^2)
	\nonumber
	\\
	u_{\theta}^{\epsilon(-)}(R,\theta)=&-u_{x_{1}}(\bm{x}_{0})\sin(\theta)+u_{x_{2}}(\bm{x}_{0})\cos(\theta)
	+\epsilon\hat{w}_{\theta}^{(I\hspace{-.15em}I-)}(r,\theta)+O(\epsilon^2)
	\nonumber
	\\
	u_{R}^{\epsilon(+)}(R,\theta)=&u_{x_{1}}(\bm{x})\cos(\theta)+u_{x_{2}}(\bm{x})\sin(\theta)
	+\epsilon\hat{w}_{r}^{(I\hspace{-.15em}I+)}(r,\theta)+O(\epsilon^2)
	\nonumber
	\\
	u_{\theta}^{\epsilon(+)}(R,\theta)=&-u_{x_{1}}(\bm{x})\sin(\theta)+u_{x_{2}}(\bm{x})\cos(\theta)
	+\epsilon\hat{w}_{\theta}^{(I\hspace{-.15em}I+)}(r,\theta)+O(\epsilon^2)
	\label{eq:SDForTD}
\end{align}
歪みは以下のようになる.
\begin{align}
	e_{RR}^{\epsilon(-)}
		=&\frac{\partial u_{R}^{\epsilon(-)}}{\partial R}
		\nonumber
		\\
		=&\epsilon\frac{1}{\epsilon}\frac{\partial\hat{w}_{r}^{(I\hspace{-.15em}I-)}}{\partial r}(r,\theta)+O(\epsilon)
		\nonumber
		\\
		=&\frac{\partial\hat{w}_{r}^{(I\hspace{-.15em}I-)}}{\partial r}(r,\theta)+O(\epsilon)
		\label{eq:eRRInEps}
\end{align}
\begin{align}
	e_{R\theta}^{\epsilon(-)}
		=&\frac{1}{2}\Bigl( \frac{1}{R}\frac{\partial u_{R}^{\epsilon(-)}}{\partial \theta}
			+\frac{\partial u_{\theta}^{\epsilon(-)}}{\partial R}-\frac{u_{\theta}^{\epsilon(-)}}{R}\Bigr)
		\nonumber
		\\
		=&\frac{1}{2}\Bigl( \frac{1}{r}\frac{\partial \hat{w}_{r}^{(I\hspace{-.15em}I-)}}{\partial \theta}
			+\frac{\partial \hat{w}_{\theta}^{(I\hspace{-.15em}I-)}}{\partial r}
			-\frac{\hat{w}_{\theta}^{(I\hspace{-.15em}I-)}}{r}\Bigr)+O(\epsilon)
\end{align}
\begin{align}
	e_{\theta\theta}^{\epsilon(-)}
		=&\frac{1}{R}\frac{\partial u_{\theta}^{\epsilon(-)}}{\partial \theta}
			+\frac{u_{R}^{\epsilon(-)}}{R}
		\nonumber
		\\
		=&\frac{1}{r}\frac{\partial \hat{w}_{\theta}^{(I\hspace{-.15em}I-)}}{\partial \theta}
			+\frac{\hat{w}_{r}^{(I\hspace{-.15em}I-)}}{r}+O(\epsilon)
\end{align}
\begin{align}
	e_{RR}^{\epsilon(+)}
		=&\frac{\partial u_{R}^{\epsilon(+)}}{\partial R}
		\nonumber
		\\
		=&\Bigl\{ \frac{\partial u_{x_{1}}}{\partial x_{1}}(\bm{x})\cos(\theta)
			+\frac{\partial u_{x_{1}}}{\partial x_{2}}(\bm{x})\sin(\theta)\Bigr\}\cos(\theta)
		\nonumber
		\\
		&+\Bigl\{ \frac{\partial u_{x_{2}}}{\partial x_{1}}(\bm{x})\cos(\theta)
			+\frac{\partial u_{x_{2}}}{\partial x_{2}}(\bm{x})\sin(\theta)\Bigr\}\sin(\theta)
		+\epsilon\frac{1}{\epsilon}\frac{\partial\hat{w}_{r}^{(I\hspace{-.15em}I+)}}{\partial r}(r,\theta)+O(\epsilon)
		\nonumber
		\\
		=&\frac{1}{2}\Bigl\{ \frac{\partial u_{x_{1}}}{\partial x_{1}}(\bm{x})
			+\frac{\partial u_{x_{2}}}{\partial x_{2}}(\bm{x})\Bigr\}
		+\frac{1}{2}\Bigl\{ \frac{\partial u_{x_{1}}}{\partial x_{1}}(\bm{x})
			-\frac{\partial u_{x_{2}}}{\partial x_{2}}(\bm{x})\Bigr\}\cos(2\theta)
		\nonumber
		\\
		&+\frac{1}{2}\Bigl\{ \frac{\partial u_{x_{1}}}{\partial x_{2}}(\bm{x})
			+\frac{\partial u_{x_{2}}}{\partial x_{1}}(\bm{x})\Bigr\}\sin(2\theta)
		+\frac{\partial\hat{w}_{r}^{(I\hspace{-.15em}I+)}}{\partial r}(r,\theta)+O(\epsilon)
		\label{eq:eRROutEps}
\end{align}
\begin{align}
	e_{R\theta}^{\epsilon(+)}
		=&\frac{1}{2}\Bigl( \frac{1}{R}\frac{\partial u_{R}^{\epsilon(+)}}{\partial \theta}
			+\frac{\partial u_{\theta}^{\epsilon(+)}}{\partial R}-\frac{u_{\theta}^{\epsilon(+)}}{R}\Bigr)
		\nonumber
		\\
		=&\frac{1}{2}\Bigl[ \frac{1}{R}\bigl\{-u_{x1}(\bm{x})\sin(\theta)
				+u_{x2}(\bm{x})\cos(\theta)\bigr\}
			+\Bigl\{ -\frac{\partial u_{x_{1}}}{\partial x_{1}}(\bm{x})\sin(\theta)
				+\frac{\partial u_{x_{1}}}{\partial x_{2}}(\bm{x})\cos(\theta)\Bigr\}\cos(\theta)
			\nonumber
			\\
			&+\Bigl\{ -\frac{\partial u_{x_{2}}}{\partial x_{1}}(\bm{x})\sin(\theta)
				+\frac{\partial u_{x_{2}}}{\partial x_{2}}(\bm{x})\cos(\theta)\Bigr\}\sin(\theta)
			-\Bigl\{ \frac{\partial u_{x_{1}}}{\partial x_{1}}(\bm{x})\cos(\theta)
				+\frac{\partial u_{x_{1}}}{\partial x_{2}}(\bm{x})\sin(\theta)\Bigr\}\sin(\theta)
			\nonumber
			\\
			&+\Bigl\{ \frac{\partial u_{x_{2}}}{\partial x_{1}}(\bm{x})\cos(\theta)
				+\frac{\partial u_{x_{2}}}{\partial x_{2}}(\bm{x})\sin(\theta)\Bigr\}\cos(\theta)
			-\frac{1}{R}\bigl\{-u_{x1}(\bm{x})\sin(\theta)+u_{x2}(\bm{x})\cos(\theta)\bigr\}\Bigr]
			\nonumber
			\\
		&+\frac{1}{2}\Bigl( \frac{1}{r}\frac{\partial \hat{w}_{r}^{(I\hspace{-.15em}I-)}}{\partial \theta}
			+\frac{\partial \hat{w}_{\theta}^{(I\hspace{-.15em}I-)}}{\partial r}
			-\frac{\hat{w}_{\theta}^{(I\hspace{-.15em}I-)}}{r}\Bigr)+O(\epsilon)
		\nonumber
		\\
		=&-\frac{1}{2}\Bigl\{ \frac{\partial u_{x_{1}}}{\partial x_{1}}(\bm{x})
			-\frac{\partial u_{x_{2}}}{\partial x_{2}}(\bm{x})\Bigr\}\sin(2\theta)
		+\frac{1}{2}\Bigl\{ \frac{\partial u_{x_{1}}}{\partial x_{2}}(\bm{x})
			+\frac{\partial u_{x_{2}}}{\partial x_{1}}(\bm{x})\Bigr\}\cos(2\theta)
			\nonumber
			\\
		&+\frac{1}{2}\Bigl( \frac{1}{r}\frac{\partial \hat{w}_{r}^{(I\hspace{-.15em}I+)}}{\partial \theta}
			+\frac{\partial \hat{w}_{\theta}^{(I\hspace{-.15em}I+)}}{\partial r}
			-\frac{\hat{w}_{\theta}^{(I\hspace{-.15em}I+)}}{r}\Bigr)+O(\epsilon)
\end{align}
\begin{align}
	e_{\theta\theta}^{\epsilon(+)}
		=&\frac{1}{R}\frac{\partial u_{\theta}^{\epsilon(+)}}{\partial \theta}
			+\frac{u_{R}^{\epsilon(+)}}{R}
		\nonumber
		\\
		=&\frac{1}{R}\bigl\{-u_{x_{1}}(\bm{x})\cos(\theta)-u_{x_{2}}(\bm{x})\sin(\theta)\bigr\}
		-\Bigl\{ -\frac{\partial u_{x_{1}}}{\partial x_{1}}(\bm{x})\sin(\theta)
			+\frac{\partial u_{x_{1}}}{\partial x_{2}}(\bm{x})\cos(\theta)\Bigr\}\sin(\theta)
		\nonumber
		\\
		&+\Bigl\{ -\frac{\partial u_{x_{2}}}{\partial x_{1}}(\bm{x})\sin(\theta)
			+\frac{\partial u_{x_{2}}}{\partial x_{2}}(\bm{x})\cos(\theta)\Bigr\}\cos(\theta)
		+\frac{1}{R}\bigl\{u_{x_{1}}(\bm{x})\cos(\theta)u_{x_{2}}(\bm{x})\sin(\theta)\bigr\}
		\nonumber
		\\
		=&\frac{1}{2}\Bigl\{ \frac{\partial u_{x_{1}}}{\partial x_{1}}(\bm{x})
			+\frac{\partial u_{x_{2}}}{\partial x_{2}}(\bm{x})\Bigr\}
		-\frac{1}{2}\Bigl\{ \frac{\partial u_{x_{1}}}{\partial x_{1}}(\bm{x})
			-\frac{\partial u_{x_{2}}}{\partial x_{2}}(\bm{x})\Bigr\}\cos(2\theta)
		\nonumber
		\\
		&-\frac{1}{2}\Bigl\{ \frac{\partial u_{x_{1}}}{\partial x_{2}}(\bm{x})
			+\frac{\partial u_{x_{2}}}{\partial x_{1}}(\bm{x})\Bigr\}\sin(2\theta)
		+\frac{1}{r}\frac{\partial \hat{w}_{\theta}^{(I\hspace{-.15em}I+)}}{\partial \theta}
			+\frac{\hat{w}_{r}^{(I\hspace{-.15em}I+)}}{r}+O(\epsilon)
	\label{eq:eThThOutEps}
\end{align}

$\Gamma_\epsilon$上での歪みは以下のようになる.
\begin{align}
	e_{RR}^{\epsilon(-)}
	=&\frac{\mu^{a}\kappa^{a}}{2\bigl(\mu^{a}(\kappa^{b}-1)+2\mu^{b}\bigr)}
	\bigl\{u_{x_{1},x_{1}}(\bm{x_{0}})+u_{x_{2},x_{2}}(\bm{x_{0}})\bigr\}
	\nonumber
	\\
	&+\frac{\mu^{a}\bigl(\kappa^{a}+1\bigr)}{2\bigl(\mu^{a}+\kappa^{a}\mu^{b}\bigr)}
	\bigl\{u_{x_{1},x_{1}}(\bm{x_{0}})-u_{x_{2},x_{2}}(\bm{x_{0}})\bigr\}\cos(2\theta)
	\nonumber
	\\
	&+\frac{\mu^{a}\bigl(\kappa^{a}+1\bigr)}{2\bigl(\mu^{a}+\kappa^{a}\mu^{b}\bigr)}
	\bigl\{u_{x_{1},x_{2}}(\bm{x_{0}})+u_{x_{2},x_{1}}(\bm{x_{0}})\bigr\}\sin(2\theta)
	\label{eq:eRRInEpsSol}
\end{align}
\begin{align}
	e_{R\theta}^{\epsilon(-)}
	=&-\frac{\mu^{a}\bigl(\kappa^{a}+1\bigr)}{2\bigl(\mu^{a}+\kappa^{a}\mu^{b}\bigr)}
	\bigl\{u_{x_{1},x_{1}}(\bm{x_{0}})-u_{x_{2},x_{2}}(\bm{x_{0}})\bigr\}\sin(2\theta)
	\nonumber
	\\
	&+\frac{\mu^{a}\bigl(\kappa^{a}+1\bigr)}{2\bigl(\mu^{a}+\kappa^{a}\mu^{b}\bigr)}
	\bigl\{u_{x_{1},x_{2}}(\bm{x_{0}})+u_{x_{2},x_{1}}(\bm{x_{0}})\bigr\}\cos(2\theta)
	\label{eq:eRThInEpsSol}
\end{align}
\begin{align}
	e_{\theta\theta}^{\epsilon(-)}
	=&\frac{\mu^{a}\kappa^{a}}{2\bigl(\mu^{a}(\kappa^{b}-1)+2\mu^{b}\bigr)}
	\bigl\{u_{x_{1},x_{1}}(\bm{x_{0}})+u_{x_{2},x_{2}}(\bm{x_{0}})\bigr\}
	\nonumber
	\\
	&-\frac{\mu^{a}\bigl(\kappa^{a}+1\bigr)}{2\bigl(\mu^{a}+\kappa^{a}\mu^{b}\bigr)}
	\bigl\{u_{x_{1},x_{1}}(\bm{x_{0}})-u_{x_{2},x_{2}}(\bm{x_{0}})\bigr\}\cos(2\theta)
	\nonumber
	\\
	&-\frac{\mu^{a}\bigl(\kappa^{a}+1\bigr)}{2\bigl(\mu^{a}+\kappa^{a}\mu^{b}\bigr)}
	\bigl\{u_{x_{1},x_{2}}(\bm{x_{0}})+u_{x_{2},x_{1}}(\bm{x_{0}})\bigr\}\sin(2\theta)
	\label{eq:eThThInEpsSol}
\end{align}

%外側の歪み
\begin{align}
	e_{RR}^{\epsilon(+)}
	=&\frac{ \mu^{a}\bigl(\kappa^{b}-1\bigr)\bigl(\kappa^{a}-2\bigr)+4\mu^{b}\bigl(\kappa^{a}-1\bigr) }
		{2\bigl(\mu^{a}(\kappa^{b}-1)+2\mu^{b}\bigr)(\kappa^{a}-1)}
	\bigl\{u_{x_{1},x_{1}}(\bm{x_{0}})+u_{x_{2},x_{2}}(\bm{x_{0}})\bigr\}
	\nonumber
	\\
	&+\frac{\mu^{a}\bigl(3-\kappa^{a}\bigr)+2\mu^{b}\bigl(\kappa^{a}-1\bigr)}{2\bigl(\mu^{a}+\kappa^{a}\mu^{b}\bigr)}
	\bigl\{u_{x_{1},x_{1}}(\bm{x_{0}})-u_{x_{2},x_{2}}(\bm{x_{0}})\bigr\}\cos(2\theta)
	\nonumber
	\\
	&+\frac{\mu^{a}\bigl(3-\kappa^{a}\bigr)+2\mu^{b}\bigl(\kappa^{a}-1\bigr)}{2\bigl(\mu^{a}+\kappa^{a}\mu^{b}\bigr)}
	\bigl\{u_{x_{1},x_{2}}(\bm{x_{0}})+u_{x_{2},x_{1}}(\bm{x_{0}})\bigr\}\sin(2\theta)
	\label{eq:eRROutEpsSol}
\end{align}
\begin{align}
	e_{R\theta}^{\epsilon(+)}
	=&\frac{\mu^{a}-\mu^{b}\bigl(\kappa^{a}+2\bigr)}{\mu^{a}+\kappa^{a}\mu^{b}}
	\bigl\{u_{x_{1},x_{1}}(\bm{x_{0}})-u_{x_{2},x_{2}}(\bm{x_{0}})\bigr\}\sin(2\theta)
	\nonumber
	\\
	&-\frac{\mu^{a}-\mu^{b}\bigl(\kappa^{a}+2\bigr)}{\mu^{a}+\kappa^{a}\mu^{b}}
	\bigl\{u_{x_{1},x_{2}}(\bm{x_{0}})+u_{x_{2},x_{1}}(\bm{x_{0}})\bigr\}\cos(2\theta)
	\label{eq:eRThOutEpsSol}
\end{align}
\begin{align}
	e_{\theta\theta}^{\epsilon(+)}
	=&\frac{\mu^{a}\kappa^{a}(\kappa^{b}-1)}
	{2\bigl(\mu^{a}(\kappa^{b}-1)+2\mu^{b}\bigr)(\kappa^{a}-1)}
	\bigl\{u_{x_{1},x_{1}}(\bm{x_{0}})+u_{x_{2},x_{2}}(\bm{x_{0}})\bigr\}
	\nonumber
	\\
	&-\frac{\mu^{a}\bigl(\kappa^{a}+1\bigr)-2\kappa^{a}\mu^{b}}{2\bigl(\mu^{a}+\kappa^{a}\mu^{b}\bigr)}
	\bigl\{u_{x_{1},x_{1}}(\bm{x_{0}})-u_{x_{2},x_{2}}(\bm{x_{0}})\bigr\}\cos(2\theta)
	\nonumber
	\\
	&-\frac{\mu^{a}\bigl(\kappa^{a}+1\bigr)-2\kappa^{a}\mu^{b}}{2\bigl(\mu^{a}+\kappa^{a}\mu^{b}\bigr)}
	\bigl\{u_{x_{1},x_{2}}(\bm{x_{0}})+u_{x_{2},x_{1}}(\bm{x_{0}})\bigr\}\sin(2\theta)
	\label{eq:eThThOutEpsSol}
\end{align}
歪みと応力の関係\eqref{eq:Constitute2}より,応力は以下のようになる.
\begin{align}
	\sigma_{RR}
	=&\frac{\kappa+1}{\kappa-1}\mu e_{RR}^{}+\frac{3-\kappa}{\kappa-1}\mu e_{\theta\theta}^{}
	\nonumber
	\\
	\sigma_{\theta\theta}
	=&\frac{\kappa+1}{\kappa-1}\mu e_{\theta\theta}^{}+\frac{3-\kappa}{\kappa-1}\mu e_{RR}^{}
	\nonumber
	\\
	\sigma_{R\theta}=&2\mu e_{R\theta}^{}
	\label{eq:Constitute2}
\end{align}

\begin{align}
	\sigma_{RR}^{\epsilon(-)} =
	&\frac{2\mu^{a}\mu^{b}\kappa^{a}}{\bigl(\mu^{a}(\kappa^{b}-1)+2\mu^{b}\bigr)(\kappa^{a}-1)}
	\bigl\{u_{x_{1},x_{1}}(\bm{x_{0}})+u_{x_{2},x_{2}}(\bm{x_{0}})\bigr\}
	\nonumber
	\\
	&+\frac{\mu^{a}\mu^{b}\bigl(\kappa^{a}+1\bigr)}{\mu^{a}+\kappa^{a}\mu^{b}}
	\bigl\{u_{x_{1},x_{1}}(\bm{x_{0}})-u_{x_{2},x_{2}}(\bm{x_{0}})\bigr\}\cos(2\theta)
	\nonumber
	\\
	&+\frac{\mu^{a}\mu^{b}\bigl(\kappa^{a}+1\bigr)}{\mu^{a}+\kappa^{a}\mu^{b}}
	\bigl\{u_{x_{1},x_{2}}(\bm{x_{0}})+u_{x_{2},x_{1}}(\bm{x_{0}})\bigr\}\sin(2\theta)
	\label{eq:SigmaRRInEpsSol}
\end{align}
\begin{align}
	\sigma_{R\theta}^{\epsilon(-)} =
	&-\frac{\mu^{a}\mu^{b}\bigl(\kappa^{a}+1\bigr)}{\mu^{a}+\kappa^{a}\mu^{b}}
	\bigl\{u_{x_{1},x_{1}}(\bm{x_{0}})-u_{x_{2},x_{2}}(\bm{x_{0}})\bigr\}\sin(2\theta)
	\nonumber
	\\
	&+\frac{\mu^{a}\mu^{b}\bigl(\kappa^{a}+1\bigr)}{\mu^{a}+\kappa^{a}\mu^{b}}
	\bigl\{u_{x_{1},x_{2}}(\bm{x_{0}})+u_{x_{2},x_{1}}(\bm{x_{0}})\bigr\}\cos(2\theta)
	\label{eq:SigmaRThInEpsSol}
\end{align}
\begin{align}
	\sigma_{\theta\theta}^{\epsilon(-)} =
	&\frac{2\mu^{a}\mu^{b}\kappa^{a}}{\bigl(\mu^{a}(\kappa^{b}-1)+2\mu^{b}\bigr)(\kappa^{a}-1)}
	\bigl\{u_{x_{1},x_{1}}(\bm{x_{0}})+u_{x_{2},x_{2}}(\bm{x_{0}})\bigr\}
	\nonumber
	\\
	&-\frac{\mu^{a}\mu^{b}\bigl(\kappa^{a}+1\bigr)}{\mu^{a}+\kappa^{a}\mu^{b}}
	\bigl\{u_{x_{1},x_{1}}(\bm{x_{0}})-u_{x_{2},x_{2}}(\bm{x_{0}})\bigr\}\cos(2\theta)
	\nonumber
	\\
	&-\frac{\mu^{a}\mu^{b}\bigl(\kappa^{a}+1\bigr)}{\mu^{a}+\kappa^{a}\mu^{b}}
	\bigl\{u_{x_{1},x_{2}}(\bm{x_{0}})+u_{x_{2},x_{1}}(\bm{x_{0}})\bigr\}\sin(2\theta)
	\label{eq:SigmaThThInEpsSol}
\end{align}

\begin{align}
	\sigma_{RR}^{\epsilon(+)}
	=&\frac{ \mu^{a}\{\mu^{a}\bigl(\kappa^{b}-1\bigr)+2\mu^{b}\bigl(\kappa^{a}+1\bigr)\} }
		{\bigl(\mu^{a}(\kappa^{b}-1)+2\mu^{b}\bigr)(\kappa^{a}-1)}
	\bigl\{u_{x_{1},x_{1}}(\bm{x_{0}})+u_{x_{2},x_{2}}(\bm{x_{0}})\bigr\}
	\nonumber
	\\
	&+\frac{\mu^{a}\bigl(3-\kappa^{a}\bigr)+2\mu^{b}\bigl(\kappa^{a}-1\bigr)}{2\bigl(\mu^{a}+\kappa^{a}\mu^{b}\bigr)}
	\bigl\{u_{x_{1},x_{1}}(\bm{x_{0}})-u_{x_{2},x_{2}}(\bm{x_{0}})\bigr\}\cos(2\theta)
	\nonumber
	\\
	&+\frac{\mu^{a}\bigl(3-\kappa^{a}\bigr)+2\mu^{b}\bigl(\kappa^{a}-1\bigr)}{2\bigl(\mu^{a}+\kappa^{a}\mu^{b}\bigr)}
	\bigl\{u_{x_{1},x_{2}}(\bm{x_{0}})+u_{x_{2},x_{1}}(\bm{x_{0}})\bigr\}\sin(2\theta)
	\label{eq:eRROutEpsSol}
\end{align}
\begin{align}
	\sigma_{R\theta}^{\epsilon(+)}
	=&\frac{\mu^{a}-\mu^{b}\bigl(\kappa^{a}+2\bigr)}{\mu^{a}+\kappa^{a}\mu^{b}}
	\bigl\{u_{x_{1},x_{1}}(\bm{x_{0}})-u_{x_{2},x_{2}}(\bm{x_{0}})\bigr\}\sin(2\theta)
	\nonumber
	\\
	&-\frac{\mu^{a}-\mu^{b}\bigl(\kappa^{a}+2\bigr)}{\mu^{a}+\kappa^{a}\mu^{b}}
	\bigl\{u_{x_{1},x_{2}}(\bm{x_{0}})+u_{x_{2},x_{1}}(\bm{x_{0}})\bigr\}\cos(2\theta)
	\label{eq:eRThOutEpsSol}
\end{align}
\begin{align}
	\sigma_{\theta\theta}^{\epsilon(+)}
	=&\frac{\mu^{a}\kappa^{a}(\kappa^{b}-1)}
	{2\bigl(\mu^{a}(\kappa^{b}-1)+2\mu^{b}\bigr)(\kappa^{a}-1)}
	\bigl\{u_{x_{1},x_{1}}(\bm{x_{0}})+u_{x_{2},x_{2}}(\bm{x_{0}})\bigr\}
	\nonumber
	\\
	&-\frac{\mu^{a}\bigl(\kappa^{a}+1\bigr)-2\kappa^{a}\mu^{b}}{2\bigl(\mu^{a}+\kappa^{a}\mu^{b}\bigr)}
	\bigl\{u_{x_{1},x_{1}}(\bm{x_{0}})-u_{x_{2},x_{2}}(\bm{x_{0}})\bigr\}\cos(2\theta)
	\nonumber
	\\
	&-\frac{\mu^{a}\bigl(\kappa^{a}+1\bigr)-2\kappa^{a}\mu^{b}}{2\bigl(\mu^{a}+\kappa^{a}\mu^{b}\bigr)}
	\bigl\{u_{x_{1},x_{2}}(\bm{x_{0}})+u_{x_{2},x_{1}}(\bm{x_{0}})\bigr\}\sin(2\theta)
	\label{eq:eThThOutEpsSol}
\end{align}


\end{document}

